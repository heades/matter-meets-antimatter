\section{Proofs}
\label{sec:proofs}
\subsection{Proof of Lemma~\ref{lemma:symmetric_monoidal_monad}}
\label{subsec:proof_of_lemma:symmetric_monoidal_monad}
We denote $H;J$ by $?$ throughout the proof, but expand this
definition where necessary.  First, it is easy to see that $?$ is
symmetric monoidal, because it is the composition of two symmetric
monoidal functors.  Thus, the following diagrams all hold:
\begin{mathpar}
  \bfig
  \vSquares|ammmmma|/->`->`->``->`->`->/[
    (\wn A \oplus \wn B) \oplus \wn C`
    \wn A \oplus (\wn B \oplus \wn C)`
    \wn (A \oplus B) \oplus \wn C`
    \wn A \oplus \wn (B \oplus C)`
    \wn ((A \oplus B) \oplus C)`
    \wn (A \oplus (B \oplus C));
        {\alpha}_{\wn A,\wn B,\wn C}`
        \r{A,B} \oplus \id_{\wn C}`
        \id_{\wn A} \oplus \r{B,C}``
        \r{A \oplus B,C}`
        \r{A,B \oplus C}`
        \wn {\alpha}_{A,B,C}]
  \efig
\end{mathpar}

\begin{mathpar}
  \bfig
  \square|amma|/->`->`<-`->/<1000,500>[
    \perp \oplus \wn A`
    \wn A`
    \wn \perp \oplus \wn A`
    \wn (\perp \oplus A);
        {\lambda}_{\wn A}`
        \r{\perp} \oplus \id_{\wn A}`
        \wn {\lambda}_{A}`
        \r{\perp,A}]
  \efig
  \and
  \bfig
  \square|amma|/->`->`<-`->/<1000,500>[
    \wn A \oplus \perp`
    \wn A`
    \wn A \oplus \wn \perp`
    \wn (A \oplus \perp);
        {\rho}_{\wn A}`
        \id_{\wn A} \oplus \r{\perp}`
        \wn {\rho}_{A}`
        \r{A,\perp}]
  \efig
\end{mathpar}

\begin{mathpar}
  \bfig
  \square|amma|/->`->`->`->/<1000,500>[
    \wn A \oplus \wn B`
    \wn B \oplus \wn A`
    \wn (A \oplus B)`
    \wn (B \oplus A);
        {\beta}_{\wn A,\wn B}`
        \r{A,B}`
        \r{B,A}`
        \wn {\beta}_{A,B}]
  \efig
\end{mathpar}
Next we show that $(\wn,\eta,\mu)$ defines a monad where
$\eta_A : A \mto ?A$ is the unit of the adjunction, and
$\mu_A = \func{J}\varepsilon_{\func{H}\,A} : \wn\wn A \mto \wn A$.  It
suffices to show that every diagram of
Definition~\ref{def:symm-monoidal-monad} holds.
\begin{itemize}
\item[Case.]
  $$\bfig
  \square|ammb|<600,600>[
    \wn^3 A`
    \wn^2 A`
    \wn^2 A`
    \wn A;
    \mu_{\wn A}`
    \wn\mu_A`
    \mu_A`
    \mu_A]
  \efig$$
  It suffices to show that the following diagram commutes:
  $$\bfig
  \square|ammb|<600,600>[
    \func{J}(\func{H}(\wn^2 A))`
    \func{J}(\func{H}\,\wn A)`
    \func{J}(\func{H}\,\wn A)`
    \func{J}(\func{H}\,A);
    \func{J}\varepsilon_{\func{H}\,\wn A}`
    \func{J}(\func{H}\,\mu_A)`
    \func{J}\varepsilon_{\func{H}\,A}`
    \func{J}\varepsilon_{\func{H}\,A}]
  \efig$$
  But this diagram is equivalent to the following:
  $$\bfig
  \square|ammb|<600,600>[
    \func{H}\func{JHJH} A`
    \func{H}\,\func{JH} A`
    \func{H}\,\func{JH} A`
    \func{H}\,A;
    \varepsilon_{\func{H}\,\func{JH} A}`
    \func{H}\,\func{J}\varepsilon_{\func{H}\,A}`
    \varepsilon_{\func{H}\,A}`
    \varepsilon_{\func{H}\,A}]
  \efig$$
  The previous diagram commutes by naturality of $\varepsilon$.

\item[Case.]
  $$\bfig
  \Atrianglepair/=`<-`=`->`<-/<600,600>[
    \wn A`
    \wn A`
    \wn^2 A`
    \wn A;`
    \mu_A``
    \eta_{\wn A}`
    \wn \eta_A]
  \efig$$
  It suffices to show that the following diagrams commutes:
  $$\bfig
  \Atrianglepair/=`<-`=`->`<-/<600,600>[
    JH A`
    JH A`
    JHJH A`
    JH A;`
    J\varepsilon_{HA}``
    \eta_{JH A}`
    JH \eta_A]
  \efig$$
  Both of these diagrams commute because $\eta$ and $\varepsilon$
  are the unit and counit of an adjunction.
\end{itemize}

It remains to be shown that $\eta$ and $\mu$ are both
symmetric monoidal natural transformations, but this easily follows
from the fact that we know $\eta$ is by assumption, and that $\mu$
is because it is defined in terms of $\varepsilon$ which is a
symmetric monoidal natural transformation.  Thus, all of the
following diagrams commute:
\begin{mathpar}
  \bfig
  \dtriangle|mmb|<1000,600>[
    A \oplus B`
    \wn A \oplus \wn B`
    \wn (A \oplus B);
    \eta_A \oplus \eta_B`
    \eta_A`
    \r{A,B}]    
  \efig
  \and
  \bfig
  \Vtriangle/->`=`<-/<600,600>[
    \perp`
    \wn \perp`
    \perp;
    \eta_\perp``
    \r{\perp}]
  \efig
\end{mathpar}
\begin{mathpar}
  \bfig
  \square|ammm|/->`->``/<1050,600>[
    \wn^2 A \oplus \wn ^2 B`
    \wn (\wn A \oplus \wn B)`
    \wn A \oplus \wn B`;
    \r{\wn A,\wn B}`
    \mu_A \oplus \mu_B``]

  \square(850,0)|ammm|/->``->`/<1050,600>[
    \wn (\wn A \oplus \wn B)`
    \wn^2(A \oplus B)``
    \wn (A \oplus B);
    \wn \r{A,B}``
    \mu_{A \oplus B}`]
  \morphism(-200,0)<2100,0>[\wn A \oplus \wn B`\wn (A \oplus B);\r{A,B}]
  \efig
  \and
  \bfig
  \square|ammb|/->`->`->`<-/<600,600>[
    \perp`
    \wn \perp`
    \wn \perp`
    \wn^2\perp;
    \r{\perp}`
    \r{\perp}`
    \wn \r{\perp}`
    \mu_\perp]
  \efig
\end{mathpar}
% subsection proof_of_lemma:symmetric_monoidal_monad (end)
% section proofs (end)
