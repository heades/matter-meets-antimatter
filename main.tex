\documentclass{llncs}

\usepackage{amssymb,amsmath}
\usepackage{cmll}
\usepackage{txfonts}
\usepackage{graphicx}
\usepackage{stmaryrd}
\usepackage{todonotes}
\usepackage{mathpartir}
\usepackage{hyperref}
\usepackage[barr]{xy}

%% This renames Barr's \to to \mto.  This allows us to use \to for imp
%% and \mto for a inline morphism.
\let\mto\to
\let\to\relax
\newcommand{\to}{\rightarrow}

% Commands that are useful for writing about type theory and programming language design.
%% \newcommand{\case}[4]{\text{case}\ #1\ \text{of}\ #2\text{.}#3\text{,}#2\text{.}#4}
\newcommand{\interp}[1]{\llbracket #1 \rrbracket}
\newcommand{\normto}[0]{\rightsquigarrow^{!}}
\newcommand{\join}[0]{\downarrow}
\newcommand{\redto}[0]{\rightsquigarrow}
\newcommand{\nat}[0]{\mathbb{N}}
\newcommand{\fun}[2]{\lambda #1.#2}
\newcommand{\CRI}[0]{\text{CR-Norm}}
\newcommand{\CRII}[0]{\text{CR-Pres}}
\newcommand{\CRIII}[0]{\text{CR-Prog}}
\newcommand{\subexp}[0]{\sqsubseteq}
%% Must include \usepackage{mathrsfs} for this to work.

\date{}

% Cat commands.
\newcommand{\powerset}[1]{\mathcal{P}(#1)}
\newcommand{\cat}[1]{\mathcal{#1}}
\newcommand{\catop}[1]{\cat{#1}^{\mathsf{op}}}
\newcommand{\Hom}[3]{\mathsf{Hom}_{\cat{#1}}(#2,#3)}
\newcommand{\colimp}[0]{\multimapdotinv}
\newcommand{\dial}[1]{\mathsf{Dial_{#1}}(\mathsf{Sets^{op}})}
\newcommand{\dialSets}[1]{\mathsf{Dial_{#1}}(\mathsf{Sets})}
\newcommand{\dcSets}[1]{\mathsf{DC_{#1}}(\mathsf{Sets})}
\newcommand{\sets}[0]{\mathsf{Sets}}
\newcommand{\obj}[1]{\mathsf{Obj}(#1)}
\newcommand{\mor}[1]{\mathsf{Mor(#1)}}
\newcommand{\id}[0]{\mathsf{id}}
\newcommand{\lett}[0]{\mathsf{let}\,}
\newcommand{\inn}[0]{\,\mathsf{in}\,}
\newcommand{\cur}[1]{\mathsf{cur}(#1)}
\newcommand{\curi}[1]{\mathsf{cur}^{-1}(#1)}

\newenvironment{changemargin}[2]{%
  \begin{list}{}{%
    \setlength{\topsep}{0pt}%
    \setlength{\leftmargin}{#1}%
    \setlength{\rightmargin}{#2}%
    \setlength{\listparindent}{\parindent}%
    \setlength{\itemindent}{\parindent}%
    \setlength{\parsep}{\parskip}%
  }%
  \item[]}{\end{list}}

%% Ott
\input{atrees-ott}

\begin{document}

\title{A Model of Co-intuitionistic Linear Logic in Dialectica Categories}

\author{Harley Eades III}
\institute{Computer and Information Sciences, Augusta University, Augusta, GA, \email{heades@augusta.edu}}

\maketitle 

\begin{abstract}
  TODO
\end{abstract}

\section{Introduction}
\label{sec:introduction}

TODO \cite{?}
% section introduction (end)

\section{Symmetric Monoidal Co-closed Categories}
\label{sec:symmetric_monoidal_co-closed_categories}

In this section we recall the definition of a symmetric monoidal
category \cite{?}, and perhaps the lesser known definition of a
symmetric monoidal left-closed category \cite{?}.  

\begin{definition}
  \label{def:monoidal-category}
  A \textbf{symmetric monoidal category (SMC)} is a category, $\cat{M}$,
  with the following data:
  \begin{itemize}
  \item An object $I$ of $\cat{M}$,
  \item A bi-functor $\oplus : \cat{M} \times \cat{M} \mto \cat{M}$,
  \item The following natural isomorphisms:
    \[
    \begin{array}{lll}
      \lambda_A : I \oplus A \mto A\\
      \rho_A : A \oplus I \mto A\\      
      \alpha_{A,B,C} : (A \oplus B) \oplus C \mto A \oplus (B \oplus C)\\
    \end{array}
    \]
  \item A symmetry natural transformation:
    \[
    \beta_{A,B} : A \oplus B \mto B \oplus A
    \]
  \item Subject to the following coherence diagrams:
    \begin{mathpar}
      \bfig
      \vSquares|ammmmma|/->`->```->``<-/[
        ((A \oplus B) \oplus C) \oplus D`
        (A \oplus (B \oplus C)) \oplus D`
        (A \oplus B) \oplus (C \oplus D)``
        A \oplus (B \oplus (C \oplus D))`
        A \oplus ((B \oplus C) \oplus D);
        \alpha_{A,B,C} \oplus \id_D`
        \alpha_{A \oplus B,C,D}```
        \alpha_{A,B,C \oplus D}``
        \id_A \oplus \alpha_{B,C,D}]      
      
      \morphism(1433,1000)|m|<0,-1000>[
        (A \oplus (B \oplus C)) \oplus D`
        A \oplus ((B \oplus C) \oplus D);
        \alpha_{A,B \oplus C,D}]
      \efig
      \and
      \bfig
      \hSquares|aammmaa|/->`->`->``->`->`->/[
        (A \oplus B) \oplus C`
        A \oplus (B \oplus C)`
        (B \oplus C) \oplus A`
        (B \oplus A) \oplus C`
        B \oplus (A \oplus C)`
        B \oplus (C \oplus A);
        \alpha_{A,B,C}`
        \beta_{A,B \oplus C}`
        \beta_{A,B} \oplus \id_C``
        \alpha_{B,C,A}`
        \alpha_{B,A,C}`
        \id_B \oplus \beta_{A,C}]
      \efig      
    \end{mathpar}
    \begin{mathpar}
      \bfig
      \Vtriangle[
        (A \oplus I) \oplus B`
        A \oplus (I \oplus B)`
        A \oplus B;
        \alpha_{A,I,B}`
        \rho_{A}`
        \lambda_{B}]
      \efig
      \and
      \bfig
      \btriangle[
        A \oplus B`
        B \oplus A`
        A \oplus B;
        \beta_{A,B}`
        \id_{A \oplus B}`
        \beta_{B,A}]
      \efig
      \and
      \bfig
      \Vtriangle[
        I \oplus A`
        A \oplus I`
        A;
        \beta_{I,A}`
        \lambda_A`
        \rho_A]
      \efig
    \end{mathpar}    
  \end{itemize}
\end{definition}

\begin{definition}
  \label{def:SMLCC}
  A \textbf{symmetric monoidal left closed category (SMLCC)} is a symmetric monoidal category 
  $(\cat{C}, \oplus, I, \alpha_{A,B,C}, \lambda_{A}, \rho_{A}, \beta_{A,B})$
  such that
  the for any object $B$ of $\cat{C}$ the functor $- \oplus B : \cat{C} \to \cat{C}$ has a specified 
  left adjoint $B \multimapdotinv - : \cat{C} \to \cat{C}$.  This means that for any objects
  $A$,$B$, and $C$ of $\cat{C}$, we have the following bijection:
  \begin{center}
    \begin{math}
      \Hom{C}{C}{A \oplus B} \cong \Hom{C}{B \multimapdotinv C}{A}
    \end{math}
  \end{center}
  that is natural in all arguments.  
\end{definition}

As we can see from the definition a SMLCC is the categorical dual to
symmetric monoidal closed categories.  Just as SMCCs model
intuitionistic linear logics \cite{?} SMLCCs model co-intuitionistic
linear logics \cite{Bellin:2015}.  There are many concrete examples of
SMLCCs, in fact, the dual of any interesting concrete SMCCs must be an
interesting SMLCC.

In this paper I make use of a specific SMLCC, $\mathsf{Set^{op}}$,
which is the dual of the category $\mathsf{Set}$ of all sets and
functions between them.  The definition of $\mathsf{Set^{op}}$ I use
here is the well-known one in power-set algebras
$(\powerset{X},\cup,\cap,\overline{\stackrel{\,\,\,\,}{\stackrel{\,\,\,\,}{\,\,\,\,}}},\emptyset,X)$
where $X$ is a set.  In fact, it is well-known that we can define the functor $\mathcal{P} : \mathsf{Set} \to \mathsf{Set^{op}}$
on morphisms by $\mathcal{P}(f : X \to Y)(S \in \powerset{Y}) = \{ x \in X \mid f(x) \in S \}$.   Thus,
objects of $\mathsf{Set^{op}}$ are powersets and morphisms are set theoretic functions between powersets.
% section symmetric_monoidal_co-closed_categories (end)

\section{The category $\dial{L}$}
\label{sec:the_category_dialL}
Dialectica categories originate from de Paiva's thesis
\cite{dePaiva:1987}, and are one of the first models of intuitionistic
linear logic.  In fact, they are the first categorical model of full
intuitionistic linear logic which is intuitionistic linear logic with
every logical connective from linear logic; complete with multiple
conclusions.  In full generality, dialectica categories, denoted
$\mathsf{Dial}_{L}(\cat{C})$, are symmetric monoidal closed categories
constructed in terms of a lineale \cite{?}, $L$, and a symmetric
monoidal closed category $\cat{C}$.  However, I take $L$ to be a
colineale (Definition~\ref{def:colineale}) and $\cat{C}$ to be a
symmetric monoidal left-closed category (Definition~\ref{def:SMLCC}),
particularly, I take $\cat{C}$ to be $\mathsf{Set^{op}}$, but I take
care to insure that all constructions lift to the general case of an
arbitrary symmetric monoidal left-closed category.  This is the first
time this construction has been given.

We begin by introducing colineales as the categorical dual to
lineales. The following defines when a proset is symmetric monoidal.
\begin{definition}
  \label{def:monoidal-proset}
  A \textbf{monoidal proset} is a proset, $(L, \leq)$, with a given
  symmetric monoidal structure $(L, \bullet, e)$.  That is, a set $L$
  with a given binary relation $\leq : L \times L \to L$ satisfying
  the following:
  \begin{itemize}
  \item (reflexivity) $a \leq a$ for any $a \in L$
  \item (transitivity) If $a \leq b$ and $b \leq c$, then $a \leq c$
  \end{itemize}
  together with a monoidal structure $(\bullet, e)$ consisting of a
  binary operation, called multiplication, $\bullet : L \times L \to L$
  and a distinguished element $e \in L$ called the unit such that the
  following hold:
  \begin{itemize}
  \item (associativity) $(a \bullet b) \bullet c = a \bullet (b \bullet c)$
  \item (identity) $a \bullet e = a = e \bullet a$
  \item (symmetry) $a \bullet b = b \bullet a$
  \end{itemize}
  Finally, the structures must be compatible, that is, if $a \leq b$,
  then $a \bullet c \leq b \bullet c$ for any $c \in L$.
\end{definition}
A colineale is essentially a symmetric monoidal left-closed
category in the category of prosets.
\begin{definition}
  \label{def:lineale}
  A \textbf{colineale} is a monoidal proset, $(L, \leq, \bullet, e)$, with
  a given binary operation, called coimplication, $\colimp : L \times L
  \to L$, such that the following hold:
  \begin{itemize}
  \item (relative complement) $b \leq a \bullet (a \colimp b)$
  \item (adjunction) If $b \colimp c \leq a$, then $c \leq a \bullet b$
  \end{itemize}
\end{definition}
An example of a concrete colineale is the three element set
$\mathsf{3} = \{0,\perp,1\}$ where $\perp$ stands for undefined\footnote{The full
  definition of the colineale $\mathsf{3}$ can be found in the formal
  development: ?}.  However, one must be careful when defining the
colineale because it is possible to degenerate to classical logic.
% section the_category_dialL (end)


\bibliographystyle{plain} \bibliography{ref}

\end{document}

%%% Local Variables: 
%%% mode: latex
%%% TeX-master: t
%%% End: 

