\documentclass[letterpaper,USenglish]{lipics-v2016}

\usepackage{amssymb,amsmath}
\usepackage{cmll}
\usepackage{txfonts}
\usepackage{graphicx}
\usepackage{stmaryrd}
\usepackage{todonotes}
\usepackage{mathpartir}
\usepackage{hyperref}
\usepackage{mdframed}
\usepackage[barr]{xy}

%% This renames Barr's \to to \mto.  This allows us to use \to for imp
%% and \mto for a inline morphism.
\let\mto\to
\let\to\relax
\newcommand{\to}{\rightarrow}

% Commands that are useful for writing about type theory and programming language design.
%% \newcommand{\case}[4]{\text{case}\ #1\ \text{of}\ #2\text{.}#3\text{,}#2\text{.}#4}
\newcommand{\interp}[1]{\llbracket #1 \rrbracket}
\newcommand{\normto}[0]{\rightsquigarrow^{!}}
\newcommand{\join}[0]{\downarrow}
\newcommand{\redto}[0]{\rightsquigarrow}
\newcommand{\nat}[0]{\mathbb{N}}
\newcommand{\fun}[2]{\lambda #1.#2}
\newcommand{\CRI}[0]{\text{CR-Norm}}
\newcommand{\CRII}[0]{\text{CR-Pres}}
\newcommand{\CRIII}[0]{\text{CR-Prog}}
\newcommand{\subexp}[0]{\sqsubseteq}
%% Must include \usepackage{mathrsfs} for this to work.

\date{}

% Cat commands.
\newcommand{\powerset}[1]{\mathcal{P}(#1)}
\newcommand{\cat}[1]{\mathcal{#1}}
\newcommand{\catop}[1]{\cat{#1}^{\mathsf{op}}}
\newcommand{\Hom}[3]{\mathsf{Hom}_{\cat{#1}}(#2,#3)}
\newcommand{\limp}[0]{\multimap}
\newcommand{\colimp}[0]{\multimapdotinv}
\newcommand{\dial}[1]{\mathsf{Dial_{#1}}(\mathsf{Sets^{op}})}
\newcommand{\dialSets}[1]{\mathsf{Dial_{#1}}(\mathsf{Sets})}
\newcommand{\dcSets}[1]{\mathsf{DC_{#1}}(\mathsf{Sets})}
\newcommand{\sets}[0]{\mathsf{Sets}}
\newcommand{\obj}[1]{\mathsf{Obj}(#1)}
\newcommand{\mor}[1]{\mathsf{Mor(#1)}}
\newcommand{\id}[0]{\mathsf{id}}
\newcommand{\lett}[0]{\mathsf{let}\,}
\newcommand{\inn}[0]{\,\mathsf{in}\,}
\newcommand{\cur}[1]{\mathsf{cur}(#1)}
\newcommand{\curi}[1]{\mathsf{cur}^{-1}(#1)}

\newenvironment{changemargin}[2]{%
  \begin{list}{}{%
    \setlength{\topsep}{0pt}%
    \setlength{\leftmargin}{#1}%
    \setlength{\rightmargin}{#2}%
    \setlength{\listparindent}{\parindent}%
    \setlength{\itemindent}{\parindent}%
    \setlength{\parsep}{\parskip}%
  }%
  \item[]}{\end{list}}

%% Large Dot for nu:
% A larger \cdot, use with \Cdot[s] where s is a scaling factor like 1 or 2
%
% Code from: http://tex.stackexchange.com/questions/81729/a-medium-dot-in-math-mode

\newcommand*{\Cdot}[1][1.25]{%
  \mathpalette{\CdotAux{#1}}\cdot%
}
\newdimen\CdotAxis
\newcommand*{\CdotAux}[3]{%
  {%
    \settoheight\CdotAxis{$#2\vcenter{}$}%
    \sbox0{%
      \raisebox\CdotAxis{%
        \scalebox{#1}{%
          \raisebox{-\CdotAxis}{%
            $\mathsurround=0pt #2#3$%
          }%
        }%
      }%
    }%
    % Remove depth that arises from scaling.
    \dp0=0pt %
    % Decrease scaled height.
    \sbox2{$#2\bullet$}%
    \ifdim\ht2<\ht0 %
      \ht0=\ht2 %
    \fi
    % Use the same width as the original \cdot.
    \sbox2{$\mathsurround=0pt #2#3$}%
    \hbox to \wd2{\hss\usebox{0}\hss}%
  }%
}

%% Ott
\input{dtt-inc}

\title{Comonadic Matter Meets Monadic Anti-Matter: An Adjoint Model of Dualized Simple Type Theory}
\titlerunning{Comonadic Matter Meets Monadic Anti-Matter}

\author{Harley Eades III}
\affil{Computer and Information Sciences, Augusta University,
  Augusta, GA, \texttt{heades@augusta.edu}}
\authorrunning{H. Eades III}
\Copyright{Harley D. Eades III}

\subjclass{Dummy classification -- please refer to
  \url{http://www.acm.org/about/class/ccs98-html}}
\keywords{Dummy keyword -- please provide 1--5 keywords}

\begin{document}

\maketitle 

\begin{abstract}
  Bi-intuitionistic logic is a conservative extension of
  intuitionistic logic with perfect duality; every logical operator of
  the logic has a corresponding dual operator also in the logic.  This
  symmetry suggests that bi-intuitionistic logic may lead to a
  foundational theory of induction and co-induction.  Dualized Simple
  Type Theory was proposed by Eades et al. as a bi-intuitionistic type
  theory that could be used to study the relationship between
  induction and coinduction.  It is the purpose of this paper to
  complete this line of inquiry by giving a categorical model of
  Dualized Simple Type Theory using a combination of two LNL models of
  Benton, which give rise to a monadic/comonadic relationship between
  intuitionistic logic and co-intuitionistic logic.  We then extend
  both Dualized Simple Type Theory and its corresponding semantics
  with induction and co-induction.
\end{abstract}

\section{Introduction}
\label{sec:introduction}
TODO \cite{?}
% section introduction (end)

\section{Dualized Intuitionistic Linear Logic}
\label{sec:linear_dualized_type_theory}
In this section we introduce a linear version of Dualized
Intuitionistic Logic we call Dualized Intuitionistic Linear Logic
(ZILL)\footnote{I chose to not abbreviate Dualized Intuitionistic
  Linear Logic by DILL, because it is well-known to be used for Dual
  Intuitionistic Linear Logic of Barber \cite{?}}.  The novelty of the
design of ZILL is its use of two different syntactic tools to enforce
linearity and intuitionism. The former is enforced in the usual way
found in linear logic, but the latter is enforced by the layering of
the abstract Kripke graph used in DIL.  However, unlike DIL there is
no straightforward interpretation of ZILL into a relational model, but
this does not concern us here.  This formalization makes it easier to
embedded DIL into ZILL.  In addition, ZILL is a new formalization --
in fact, an extension -- of Full Intuitionistic Linear Logic (FILL)
\cite{Eades:2016}.

The syntax of ZILL is a straightforward adaption of the dualized
syntax used in the design of DIL.
\begin{definition}
  \label{def:syntax-ZILL}
  The following defines the syntax of ZILL:
  \begin{center}
    \begin{math}
      \begin{array}{lllll}
        \text{(Variance)}   & [[p]] ::= [[+]] \mid [[-]]\\
        \text{(Formulas)}   & [[A]],[[B]],[[C]] ::= [[< p >]] \mid [[A (x) p B]] \mid [[A -o p B]]\\
        \text{(Graphs)}     & [[Gr]] ::= [[.]] \mid [[n1 <=p n2]] \mid [[Gr1,Gr2]]\\
        \text{(Contexts)}   & [[G]] ::= [[.]] \mid [[p A @ n]] \mid [[G1,G2]]\\
        \text{(Sequents)}   & Q ::= [[Gr ; H |-L t : p A @ n]]
      \end{array}
    \end{math}
  \end{center}
\end{definition}
Just as with DIL variance is used to indicate when we are operating on
the left of the sequent ($[[p]] = [[-]]$) or the right of the sequent
($[[p]] = [[+]]$).  Thus, the unit of tensor is $[[<+>]]$, the unit of
par is $[[<->]]$, tensor product is denoted by $[[A (x)+ B]]$, par is
denoted by $[[A (x)- B]]$, implication is denoted by $[[A -o+ B]]$,
and finally co-implication is denoted by $[[A -o- B]]$.  The sequents
of ZILL are denoted by $[[Gr ; H |-L t : p A @ n]]$ where $A$ is called
the active formula, and $n$ is a node in the abstract Kripke model
$[[Gr]]$.

The typing rules for ZILL can be found in
Figure~\ref{fig:ZILL-typing}.
\begin{figure}
  \begin{mdframed}
    \begin{mathpar}
      \DTTdruleLXXax{} \and
      \DTTdruleLXXunit{} \and
      \DTTdruleLXXunitBar{} \and
      \DTTdruleLXXmonoL{} \and
      \DTTdruleLXXmonoR{} \and      
      \DTTdruleLXXimp{} \and
      \DTTdruleLXXimpBar{} \and
      \DTTdruleLXXcutL{} \and
      \DTTdruleLXXcutR{} \and      
      \DTTdruleLXXEx{} 
    \end{mathpar}
  \end{mdframed}
  \caption{Typing Relation for ZILL}
  \label{fig:ZILL-typing}
\end{figure}
The rules are straightforward and not very surprising.  The only thing
to note is that the cut rule found in DIL had to be split into two cut
rules in ZILL in order to maintain linearity.  The first rule binds a
free variable occurring in the first term of the cut, and the second
binds a free variable occurring in the second term of the cut. The
completeness proof of DIL heavily depends on the ability to switch out
the active formula using the left-to-right lemma (Lemma~14 of
\cite{Eades:?}) effectively modeling exchanging conclusions, but this
derivable rule depends on weakening, and hence, is no longer derivable
in ZILL, but this is expected.  Since ZILL's role is a intermediate
language between intuitionistic logic and co-intuitionistic logic this
is not a barrier.  In fact, after extending ZILL with the proper
adjoints we will regain this ability.

\subsubsection{Adjoint Extension of ZILL}
\label{subsec:adjoint_extension_of_zill}

% subsubsection adjoint_extension_of_zill (end)
% section linear_dualized_type_theory (end)


\section{Symmetric Monoidal Co-closed Categories}
\label{sec:symmetric_monoidal_co-closed_categories}

In this section we recall the definition of a symmetric monoidal
category \cite{?}, and perhaps the lesser known definition of a
symmetric monoidal left-closed category \cite{?}.  

\begin{definition}
  \label{def:monoidal-category}
  A \textbf{symmetric monoidal category (SMC)} is a category, $\cat{M}$,
  with the following data:
  \begin{itemize}
  \item An object $I$ of $\cat{M}$,
  \item A bi-functor $\oplus : \cat{M} \times \cat{M} \mto \cat{M}$,
  \item The following natural isomorphisms:
    \[
    \begin{array}{lll}
      \lambda_A : I \oplus A \mto A\\
      \rho_A : A \oplus I \mto A\\      
      \alpha_{A,B,C} : (A \oplus B) \oplus C \mto A \oplus (B \oplus C)\\
    \end{array}
    \]
  \item A symmetry natural transformation:
    \[
    \beta_{A,B} : A \oplus B \mto B \oplus A
    \]
  \item Subject to the following coherence diagrams:
    \begin{mathpar}
      \bfig
      \vSquares|ammmmma|/->`->```->``<-/[
        ((A \oplus B) \oplus C) \oplus D`
        (A \oplus (B \oplus C)) \oplus D`
        (A \oplus B) \oplus (C \oplus D)``
        A \oplus (B \oplus (C \oplus D))`
        A \oplus ((B \oplus C) \oplus D);
        \alpha_{A,B,C} \oplus \id_D`
        \alpha_{A \oplus B,C,D}```
        \alpha_{A,B,C \oplus D}``
        \id_A \oplus \alpha_{B,C,D}]      
      
      \morphism(1433,1000)|m|<0,-1000>[
        (A \oplus (B \oplus C)) \oplus D`
        A \oplus ((B \oplus C) \oplus D);
        \alpha_{A,B \oplus C,D}]
      \efig
      \and
      \bfig
      \hSquares|aammmaa|/->`->`->``->`->`->/[
        (A \oplus B) \oplus C`
        A \oplus (B \oplus C)`
        (B \oplus C) \oplus A`
        (B \oplus A) \oplus C`
        B \oplus (A \oplus C)`
        B \oplus (C \oplus A);
        \alpha_{A,B,C}`
        \beta_{A,B \oplus C}`
        \beta_{A,B} \oplus \id_C``
        \alpha_{B,C,A}`
        \alpha_{B,A,C}`
        \id_B \oplus \beta_{A,C}]
      \efig      
    \end{mathpar}
    \begin{mathpar}
      \bfig
      \Vtriangle[
        (A \oplus I) \oplus B`
        A \oplus (I \oplus B)`
        A \oplus B;
        \alpha_{A,I,B}`
        \rho_{A}`
        \lambda_{B}]
      \efig
      \and
      \bfig
      \btriangle[
        A \oplus B`
        B \oplus A`
        A \oplus B;
        \beta_{A,B}`
        \id_{A \oplus B}`
        \beta_{B,A}]
      \efig
      \and
      \bfig
      \Vtriangle[
        I \oplus A`
        A \oplus I`
        A;
        \beta_{I,A}`
        \lambda_A`
        \rho_A]
      \efig
    \end{mathpar}    
  \end{itemize}
\end{definition}

\begin{definition}
  \label{def:SMLCC}
  A \textbf{symmetric monoidal left closed category (SMLCC)} is a symmetric monoidal category 
  $(\cat{C}, \oplus, I, \alpha_{A,B,C}, \lambda_{A}, \rho_{A}, \beta_{A,B})$
  such that
  the for any object $B$ of $\cat{C}$ the functor $- \oplus B : \cat{C} \to \cat{C}$ has a specified 
  left adjoint $B \multimapdotinv - : \cat{C} \to \cat{C}$.  This means that for any objects
  $A$,$B$, and $C$ of $\cat{C}$, we have the following bijection:
  \begin{center}
    \begin{math}
      \Hom{C}{C}{A \oplus B} \cong \Hom{C}{B \multimapdotinv C}{A}
    \end{math}
  \end{center}
  that is natural in all arguments.  
\end{definition}

As we can see from the definition a SMLCC is the categorical dual to
symmetric monoidal closed categories.  Just as SMCCs model
intuitionistic linear logics \cite{?} SMLCCs model co-intuitionistic
linear logics \cite{Bellin:2015}.  There are many concrete examples of
SMLCCs, in fact, the dual of any interesting concrete SMCCs must be an
interesting SMLCC.

In this paper I make use of a specific SMLCC, $\mathsf{Set^{op}}$,
which is the dual of the category $\mathsf{Set}$ of all sets and
functions between them.  The definition of $\mathsf{Set^{op}}$ I use
here is the well-known one in power-set algebras
$(\powerset{X},\cup,\cap,\overline{\stackrel{\,\,\,\,}{\stackrel{\,\,\,\,}{\,\,\,\,}}},\emptyset,X)$
where $X$ is a set.  In fact, it is well-known that we can define the functor $\mathcal{P} : \mathsf{Set} \to \mathsf{Set^{op}}$
on morphisms by $\mathcal{P}(f : X \to Y)(S \in \powerset{Y}) = \{ x \in X \mid f(x) \in S \}$.   Thus,
objects of $\mathsf{Set^{op}}$ are powersets and morphisms are set theoretic functions between powersets.
% section symmetric_monoidal_co-closed_categories (end)

\section{The category $\dial{L}$}
\label{sec:the_category_dialL}
Dialectica categories originate from de Paiva's thesis
\cite{dePaiva:1987}, and are one of the first models of intuitionistic
linear logic.  In fact, they are the first categorical model of full
intuitionistic linear logic which is intuitionistic linear logic with
every logical connective from linear logic; complete with multiple
conclusions.  In full generality, dialectica categories, denoted
$\mathsf{Dial}_{L}(\cat{C})$, are symmetric monoidal closed categories
constructed in terms of a lineale \cite{?}, $L$, and a symmetric
monoidal closed category $\cat{C}$.  However, I take $L$ to be a
colineale (Definition~\ref{def:colineale}) and $\cat{C}$ to be a
symmetric monoidal left-closed category (Definition~\ref{def:SMLCC}),
particularly, I take $\cat{C}$ to be $\mathsf{Set^{op}}$, but I take
care to insure that all constructions lift to the general case of an
arbitrary symmetric monoidal left-closed category.  This is the first
time this construction has been given.

We begin by introducing colineales as the categorical dual to
lineales. The following defines when a proset is symmetric monoidal.
\begin{definition}
  \label{def:monoidal-proset}
  A \textbf{monoidal proset} is a proset, $(L, \leq)$, with a given
  symmetric monoidal structure $(L, \bullet, e)$.  That is, a set $L$
  with a given binary relation $\leq : L \times L \to L$ satisfying
  the following:
  \begin{itemize}
  \item (reflexivity) $a \leq a$ for any $a \in L$
  \item (transitivity) If $a \leq b$ and $b \leq c$, then $a \leq c$
  \end{itemize}
  together with a monoidal structure $(\bullet, e)$ consisting of a
  binary operation, called multiplication, $\bullet : L \times L \to L$
  and a distinguished element $e \in L$ called the unit such that the
  following hold:
  \begin{itemize}
  \item (associativity) $(a \bullet b) \bullet c = a \bullet (b \bullet c)$
  \item (identity) $a \bullet e = a = e \bullet a$
  \item (symmetry) $a \bullet b = b \bullet a$
  \end{itemize}
  Finally, the structures must be compatible, that is, if $a \leq b$,
  then $a \bullet c \leq b \bullet c$ for any $c \in L$.
\end{definition}
A colineale is essentially a symmetric monoidal left-closed
category in the category of prosets.
\begin{definition}
  \label{def:lineale}
  A \textbf{colineale} is a monoidal proset, $(L, \leq, \bullet, e)$, with
  a given binary operation, called coimplication, $\colimp : L \times L
  \to L$, such that the following hold:
  \begin{itemize}
  \item (relative complement) $b \leq a \bullet (a \colimp b)$
  \item (adjunction) If $b \colimp c \leq a$, then $c \leq a \bullet b$
  \end{itemize}
\end{definition}
An example of a concrete colineale is the three element set
$\mathsf{3} = \{0,\perp,1\}$ where $\perp$ stands for
undefined\footnote{The full definition of the colineale $\mathsf{3}$
  can be found in the formal development: ?}.  However, one must be
careful when defining the colineale $\mathsf{3}$ because it is
possible to degenerate to classical logic.

We now use colineales to define the category $\dial{L}$.  This
category is given as a construction similar to the Chu construction.
\begin{definition}
  \label{def:dialect-category}
  Suppose $(L, \leq, \bullet, e, \colimp)$ is a colineale.  Then the category $\dial{L}$ consists of 
  \begin{itemize}
  \item objects that are triples, $A = (\powerset{U},\powerset{X},\alpha)$, where $U$ and
    $X$ are sets, and $\alpha : \powerset{U} \times \powerset{X} \to L$ is a multi-relation, and
  \item maps that are pairs $(f,F) : (\powerset{U},\powerset{X},\alpha) \to (\powerset{V},\powerset{Y},\beta)$
    where $f : \powerset{U} \to \powerset{V}$ and $F : \powerset{Y} \to \powerset{X}$ such that
    \begin{itemize}
    \item For any $u \in \powerset{U}$ and $y \in \powerset{Y}$, $\alpha(u,F(y)) \leq \beta(f(u),y)$.
    \end{itemize}
  \end{itemize}
  Suppose $A = (\powerset{U},\powerset{X},\alpha)$, $B = (\powerset{V},\powerset{Y},\beta)$, and $C =
  (\powerset{W},\powerset{Z},\gamma)$.  Then identities are given by $(\id_U,\id_X) : A \to
  A$.  The composition of the maps $(f,F) : A \to B$ and $(g, G) : B
  \to C$ is defined as $(f;g,G;F) : A \to C$.
\end{definition}

% section the_category_dialL (end)


\bibliographystyle{plainurl} \bibliography{ref}

\end{document}

%%% Local Variables: 
%%% mode: latex
%%% TeX-master: t
%%% End: 

