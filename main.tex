\documentclass{lmcs}
\usepackage{amssymb,amsmath}
\usepackage{cmll}
\usepackage{txfonts}
\usepackage{graphicx}
\usepackage{stmaryrd}
\usepackage{todonotes}
\usepackage{mathpartir}
\usepackage{hyperref}
\usepackage{mdframed}
\usepackage[barr]{xy}

\newtheorem{theorem}{Theorem}
\newtheorem{lemma}[theorem]{Lemma}
\newtheorem{corollary}[theorem]{Corollary}
\newtheorem{definition}[theorem]{Definition}
\newtheorem{proposition}[theorem]{Proposition}
\newtheorem{example}[theorem]{Example}

%% This renames Barr's \to to \mto.  This allows us to use \to for imp
%% and \mto for a inline morphism.
\let\mto\to
\let\to\relax
\newcommand{\to}{\rightarrow}
\newcommand{\ndto}[1]{\to_{#1}}
\newcommand{\ndwedge}[1]{\wedge_{#1}}

% Commands that are useful for writing about type theory and programming language design.
%% \newcommand{\case}[4]{\text{case}\ #1\ \text{of}\ #2\text{.}#3\text{,}#2\text{.}#4}
\newcommand{\interp}[1]{\llbracket #1 \rrbracket}
\newcommand{\normto}[0]{\rightsquigarrow^{!}}
\newcommand{\join}[0]{\downarrow}
\newcommand{\redto}[0]{\rightsquigarrow}
\newcommand{\nat}[0]{\mathbb{N}}
\newcommand{\fun}[2]{\lambda #1.#2}
\newcommand{\CRI}[0]{\text{CR-Norm}}
\newcommand{\CRII}[0]{\text{CR-Pres}}
\newcommand{\CRIII}[0]{\text{CR-Prog}}
\newcommand{\subexp}[0]{\sqsubseteq}
%% Must include \usepackage{mathrsfs} for this to work.

\date{}

% Cat commands.
\newcommand{\powerset}[1]{\mathcal{P}(#1)}
\newcommand{\cat}[1]{\mathcal{#1}}
\newcommand{\catop}[1]{\cat{#1}^{\mathsf{op}}}
\newcommand{\Hom}[3]{\mathsf{Hom}_{\cat{#1}}(#2,#3)}
\newcommand{\limp}[0]{\multimap}
\newcommand{\colimp}[0]{\multimapdotinv}
\newcommand{\dial}[1]{\mathsf{Dial_{#1}}(\mathsf{Sets^{op}})}
\newcommand{\dialSets}[1]{\mathsf{Dial_{#1}}(\mathsf{Sets})}
\newcommand{\dcSets}[1]{\mathsf{DC_{#1}}(\mathsf{Sets})}
\newcommand{\sets}[0]{\mathsf{Sets}}
\newcommand{\obj}[1]{\mathsf{Obj}(#1)}
\newcommand{\mor}[1]{\mathsf{Mor(#1)}}
\newcommand{\id}[0]{\mathsf{id}}
\newcommand{\lett}[0]{\mathsf{let}\,}
\newcommand{\inn}[0]{\,\mathsf{in}\,}
\newcommand{\cur}[1]{\mathsf{cur}(#1)}
\newcommand{\curi}[1]{\mathsf{cur}^{-1}(#1)}

\newenvironment{changemargin}[2]{%
  \begin{list}{}{%
    \setlength{\topsep}{0pt}%
    \setlength{\leftmargin}{#1}%
    \setlength{\rightmargin}{#2}%
    \setlength{\listparindent}{\parindent}%
    \setlength{\itemindent}{\parindent}%
    \setlength{\parsep}{\parskip}%
  }%
  \item[]}{\end{list}}

%% Ott
\input{BiLNL-inc}

\urldef{\mailsa}\path|{heades}@augusta.edu|

\begin{document}

\title{Comonadic Matter Meets Monadic Anti-Matter: An Adjoint Model of Bi-Intuitionistic Logic}
\author{Harley Eades III}
\email{heades@augusta.edu}
\address{Computer and Information Sciences, Augusta University, Augusta, GA}

\maketitle 

\begin{abstract}

  Bi-intuitionistic logic (BINT) is a conservative extension of
  intuitionistic logic with perfect duality.  That is, BINT contains
  the usual intuitionistic logical connectives such as true,
  conjunction, and implication, but also their duals false,
  disjunction, and co-implication. One leading question with respect
  to BINT is, what does BINT look like across the three arcs -- logic,
  typed $\lambda$-calculi, and category theory -- of the
  Curry-Howard-Lambek correspondence?  A non-trivial (does not
  degenerate to a poset) categorical model of BINT is currently an
  open problem.  It is this open problem that this paper contributes
  to by providing the first fully developed categorical model of BINT.
  It is well-known that the linear counterpart, linear BINT, of BINT
  can be modeled in a symmetric monoidal closed category equipped with
  an additional monoidal structure that models par and a specified
  left adjoint to par called linear co-implication.  We call this
  model a symmetric bi-monoidal bi-closed category. In addition, it is
  well-known that intuitionistic logic has a categorical model of
  cartesian closed categories, and their dual co-cartesian co-closed
  categories model co-intuitionistic logic.  In this paper we exploit
  Benton's beautiful LNL models of linear logic to show that these
  three models can be mixed by requiring a symmetric monoidal
  adjunction between a cartesian closed category and the symmetric
  bi-monoidal bi-closed category, in addition to a symmetric monoidal
  adjunction between a co-cartesian co-closed category and the
  symmetric bi-monoidal bi-closed category.  As a result of this
  mixture we obtain two modalities the usual comonadic of-course
  modality of linear logic, but also a monadic modality allowing for
  the embedding of co-intuitionistic logic inside of linear BINT.
  Finally, using these modalities we show that BINT intuitionistic
  logic can be soundly modeled in this new categorical model. As a
  bi-product of this model we define BiLNL logic which can be seen as
  the mixture of intuitionistic logic with co-intuitionistic logic
  inside of linear BINT.

\end{abstract}

\section{Introduction}
\label{sec:introduction}
TODO \cite{?}
% section introduction (end)

\section{Mixed Linear/Non-Linear Models of Bi-intuitionistic Logic: The Categorical Model}
\label{sec:the_categorical_model}
TODO
% section the_categorical_model (end)

\section{Mixed Linear/Non-Linear Bi-intuitionistic Logic: BiLNL Logic}
\label{sec:bilnl_logic}

Following Benton's \cite{Benton:1994} lead we can define a mixed
linear/non-linear bi-intuitionistic logic, called BiLNL logic, based
on the categorical model given in the previous section.  BiLNL logic
consists of three fragments: an intuitionistic fragment, a
co-intuitionistic fragment, and a linear bi-intuitionistic core
fragment.  This formalization allows us to have more control on the
mixture of the intuitionistic and co-intuitionistic fragments so as to
allow for a proper categorical model.  Each of the fragments are
related through a syntactic formalization of the adjoint functors from
the BiLNL model.  First, we define the syntax of BiLNL logic, and then
discuss the inference rules for each fragment.
\begin{definition}
  \label{def:BiLNL-syntax}
  The syntax for BiLNL logic is defined as follows:
  \begin{center}
    \begin{math}
      \begin{array}{rlllllllllll}
        \text{(Worlds)} & W ::= [[w1]] \mid \cdots \mid [[wi]]\\
        \text{(Graphs)} & [[Gr]] ::= [[w1 <= w2]] \mid [[Gr1 , Gr2]]\\
        \text{(Intuitionistic Formulas)} & [[X]], [[Y]], [[Z]] ::= 1
          \mid [[X x Y]] \mid [[X -> Y]] \mid [[Gf A]]\\
        \text{(Co-intuitionistic Formulas)} &  [[R]], [[S]], [[T]] ::= 0 \mid [[S + T]] \mid [[S - T]] \mid [[H A]]\\
        \text{(Linear Bi-intuitionistic Formulas)} &
             [[A]],[[B]],[[C]] ::= [[True]] \mid [[False]] \mid
             [[A (x) B]] \mid [[A (+) B]] \mid
             [[A -o B]] \mid [[A *- B]] \mid [[F X]] \mid [[J S]]\\
             \text{(Intuitionistic Contexts)}  & [[H]] ::= [[.]] \mid
                  [[X @ w]] \mid [[H1 , H2]]\\
        \text{(Co-intuitionistic Contexts)}  & [[I]] ::= [[.]] \mid
             [[R @ w]] \mid [[I1 , I2]]\\
        \text{(Linear Bi-intuitionistic Contexts)}  &
             [[G]],[[D]] ::= [[.]] \mid [[A @ w]] \mid [[G1,G2]]\\        
      \end{array}
    \end{math}
  \end{center}
  Worlds may also be denoted by (potentially subscripted) $[[n]]$,
  $[[m]]$, and $[[o]]$.

  \ \\
  \noindent
  Sequents have the following syntax:
  \begin{center}
    \begin{math}
      \begin{array}{rll}
        \text{(Intuitionistic Sequents)}   & [[Gr;H |-I X@w]]\\
        \text{(Co-intuitionistic Sequents)} & [[Gr;R@w |-C I]]\\
        \text{(LNL Bi-intuitionistic Sequents)} & [[Gr;H | G |-L D | I]]\\
      \end{array}
    \end{math}
  \end{center}
\end{definition}

The syntax of intuitionistic and co-intuitionistic formulas are
typical.  I denote co-implication by $[[S - T]]$, but all the other
connectives are the usual ones. Linear bi-intuitionistic formulas are
denoted in somewhat of a non-traditional style. I denote the unit of
tensor by $[[True]]$ instead of the usual $[[1]]$, which is the unit
of intuitionistic conjunction, in addition, I denote par by $[[A (+)
    B]]$, instead of $[[A]] \parr [[B]]$.  Lastly, I denote linear
co-implication by $[[A *- B]]$ to emphasize its duality with linear
implication $[[A -o B]]$.  Each syntactic category of formulas
contains the respective functor from the BiLNL model, and thus, we
should few $F$ and $H$ as the left adjoints to $G$ and $J$
respectively.

Formulas in each type of context are annotated with a world, and each
sequent is annotated with a graph.  These graphs are syntactic
representations of Kripke models and are used to enforce intuitionism.
They were first used in bi-intuitionistic logic by Pinto and Uustalu
\cite{Pinto:2009} to enforce intuitionism in their logic L.  In fact,
we can see the linear core of BiLNL logic as the linear version of L.
The beauty of this type of formalization and the reason why this style
of logic was used by Pinto and Uustalu is that the logic L, and as
well as BiLNL logic, are complete for cut-free bi-intuitionistic
proofs.  This was a new result of Pinto and Uustalu, because earlier
formalizations of bi-intuitionistic logic \cite{Crolard:2001} used the
Dragalin restriction \cite{Dragalin:1988} to enforce intuitionism, but
this results in a failure of cut-elimination
\cite{Schellinx:1991,Pinto:2009}.

The expert reader will notice that it is not necessary to annotate the
sequents of intuitionistic and co-intuitionistic logic with graphs and
worlds to enforce intuitionism and co-intuitionism respectively. It is
well known that restricting the right and left contexts two a single
formula enforces intuitionism and co-intuitionism respectively.
However, when mixing these two fragments with the linear core, which
requires the graphs to be intuitionistic, it is easier if they are
annotated.  If they were not, then a seemingly complex world inference
system would need to be designed to add the world constraints before
mixing with the linear core, and it is currently an open problem
whether this can be done.  Thus, with respect to intuitionistic and
co-intuitionistic logic the graph and world annotations can be seen as
book keeping.

A second fact an expert reader will notice is that Kripke models are a
relational model of intuitionistic logic, but not intuitionistic
linear logic.  This is okay, because Kripke models enforce
intuitionism and not linearity, but it is well known how to enforce
linearity syntactically in the definition of the inference rules.  It
is also well-known that even in linear logic if sequents have multiple
hypothesis and multiple conclusions the logic becomes classical.
Thus, in BiLNL logic we combine both of these tools to enforce both
intuitionism and linearity.  We can simply few the graphs as an over
approximation of the relational constraints necessary to enforce both
intuitionism and linearity.  This also makes it easier to embed Pinto
and Uustalu's logic in BiLNL logic as we will do in
Section~\ref{sec:embedding_l_in_bilnl_logic}.

Sequents for the linear core have the form $[[Gr;H | G |-L D | I]]$.
Similarly to the sequents of Benton's LNL logic \cite{Benton:1994},
each context is separated for readability, but should actually be
understood as being able to be mixed, that is, the contexts $[[H]]$
and $[[G]]$ could be a single context, and so could $[[D]]$ and
$[[I]]$.  The sequent:
\[
[[Gr]];[[X1@w1]],\ldots,[[Xi@wi]] \mid [[A1@n1]],\ldots,[[Aj@nj]]
\vdash_{\mathsf{L}} [[B1@m1]],\ldots,[[Bk@mk]] \mid [[R1@o1]],\ldots,[[Rl@ol]]
\]
will be interpreted in a BiLNL model by a morphism of the following
form:
\[
[[F X1]] \otimes \cdots \otimes [[F Xi]] \otimes [[A1]] \otimes \cdots
\otimes [[Aj]] \mto^{f} [[B1]] \oplus \cdots \oplus [[Bk]] \oplus [[J R1]]
\oplus \cdots \oplus [[J Rl]]
\]
Thus, intuitionistic formulas in the linear core can be viewed as
being under the left adjoint $[[F]]$, and co-intuitionistic formulas
as being under the right adjoint $[[J]]$.  This implies that even in
the model all types of formulas can be freely mixed.

\begin{figure}
  \begin{mdframed}
    \begin{mathpar}
      \BiLNLdruleIXXrl{} \and
      \BiLNLdruleIXXts{} \and
      \BiLNLdruleIXXid{} \and
      \BiLNLdruleIXXcut{} \and
      \BiLNLdruleIXXwk{} \and
      \BiLNLdruleIXXcr{} \and
      \BiLNLdruleIXXex{} \and                  
      \BiLNLdruleIXXmL{} \and
      \BiLNLdruleIXXmR{} \and
      \BiLNLdruleIXXtL{} \and
      \BiLNLdruleIXXtR{} \and
      \BiLNLdruleIXXpL{} \and
      \BiLNLdruleIXXpR{} \and
      \BiLNLdruleIXXIL{} \and
      \BiLNLdruleIXXIR{} \and
      \BiLNLdruleIXXgR{}
    \end{mathpar}
  \end{mdframed}
  \caption{Inference Rules for BiLNL Logic: Intuitionistic Fragment}
  \label{fig:ifr-IL}
\end{figure}

\begin{figure}
  \begin{mdframed}
    \begin{mathpar}
      \BiLNLdruleCXXrl{} \and
      \BiLNLdruleCXXts{} \and
      \BiLNLdruleCXXid{} \and
      \BiLNLdruleCXXcut{} \and
      \BiLNLdruleCXXwk{} \and
      \BiLNLdruleCXXcr{} \and
      \BiLNLdruleCXXex{} \and                  
      \BiLNLdruleCXXmL{} \and
      \BiLNLdruleCXXmR{} \and
      \BiLNLdruleCXXfL{} \and
      \BiLNLdruleCXXfR{} \and
      \BiLNLdruleCXXdL{} \and
      \BiLNLdruleCXXdR{} \and
      \BiLNLdruleCXXsL{} \and
      \BiLNLdruleCXXsR{} \and
      \BiLNLdruleCXXhL{}
    \end{mathpar}
  \end{mdframed}
  \caption{Inference Rules for BiLNL Logic: Co-intuitionistic Fragment}
  \label{fig:ifr-CL}
\end{figure}

\begin{figure}
  \begin{mdframed}
    \begin{mathpar}
      \BiLNLdruleLXXrl{} \and
      \BiLNLdruleLXXts{} \and
      \BiLNLdruleLXXmL{} \and
      \BiLNLdruleLXXmR{} \and
      \BiLNLdruleLXXImL{} \and
      \BiLNLdruleLXXCmR{}
    \end{mathpar}
  \end{mdframed}
  \caption{Inference Rules for BiLNL Logic: Abstract Kripke Graph Rules}
  \label{fig:ifr-biLNL-graph}
\end{figure}

\begin{figure}
  \begin{mdframed}
    \begin{mathpar}
      \BiLNLdruleLXXwkL{} \and
      \BiLNLdruleLXXwkR{} \and
      \BiLNLdruleLXXctrL{} \and
      \BiLNLdruleLXXctrR{} \and
      \BiLNLdruleLXXexL{} \and
      \BiLNLdruleLXXexR{} \and
      \BiLNLdruleLXXIexL{} \and
      \BiLNLdruleLXXCexL{}
    \end{mathpar}
  \end{mdframed}
  \caption{Inference Rules for BiLNL Logic: Structural Rules}
  \label{fig:ifr-biLNL-structural}
\end{figure}

\begin{figure}
  \begin{mdframed}
    \begin{mathpar}
      \BiLNLdruleLXXid{} \and
      \BiLNLdruleLXXcut{} \and
      \BiLNLdruleLXXIcut{} \and
      \BiLNLdruleLXXCcut{} 
    \end{mathpar}
  \end{mdframed}
  \caption{Inference Rules for BiLNL Logic: Identity and Cut Rules}
  \label{fig:ifr-biLNL-id-cut}
\end{figure}

\begin{figure}
  \begin{mdframed}
    \begin{mathpar}
      \BiLNLdruleLXXIL{} \and
      \BiLNLdruleLXXIR{} \and
      \BiLNLdruleLXXcL{} \and
      \BiLNLdruleLXXtL{} \and
      \BiLNLdruleLXXtR{} 
    \end{mathpar}
  \end{mdframed}
  \caption{Inference Rules for BiLNL Logic: Conjunction and Tensor Rules}
  \label{fig:ifr-biLNL-conunction-tensor}
\end{figure}

\begin{figure}
  \begin{mdframed}
    \begin{mathpar}
      \BiLNLdruleLXXflL{} \and
      \BiLNLdruleLXXflR{} \and
      \BiLNLdruleLXXdR{} \and
      \BiLNLdruleLXXpL{} \and
      \BiLNLdruleLXXpR{} 
    \end{mathpar}
  \end{mdframed}
  \caption{Inference Rules for BiLNL Logic: Disjunction and Par Rules}
  \label{fig:ifr-biLNL-disjunction-par}
\end{figure}

\begin{figure}
  \begin{mdframed}
    \begin{mathpar}
      \BiLNLdruleLXXImpL{} \and
      \BiLNLdruleLXXImpR{} \and
      \BiLNLdruleLXXIImpL{} 
    \end{mathpar}
  \end{mdframed}
  \caption{Inference Rules for BiLNL Logic: Implication Rules}
  \label{fig:ifr-biLNL-implication}
\end{figure}

\begin{figure}
  \begin{mdframed}
    \begin{mathpar}
      \BiLNLdruleLXXsL{} \and
      \BiLNLdruleLXXsR{} \and
      \BiLNLdruleLXXCsR{} 
    \end{mathpar}
  \end{mdframed}
  \caption{Inference Rules for BiLNL Logic: Co-implication Rules}
  \label{fig:ifr-biLNL-co-implication}
\end{figure}

\begin{figure}
  \begin{mdframed}
    \begin{mathpar}
      \BiLNLdruleLXXfL{} \and
      \BiLNLdruleLXXfR{} \and
      \BiLNLdruleLXXjL{} \and
      \BiLNLdruleLXXjR{} \and
      \BiLNLdruleLXXgL{} \and
      \BiLNLdruleLXXhR{} 
    \end{mathpar}
  \end{mdframed}
  \caption{Inference Rules for BiLNL Logic: Adjoint Functors Rules}
  \label{fig:ifr-biLNL-adjoint-functors}
\end{figure}
% section bilnl_logic (end)

\section{Embedding Bi-Intuitionistic Logic in BiLNL Logic}
\label{sec:embedding_l_in_bilnl_logic}

TODO
\begin{figure}
  \begin{mdframed}
    \begin{mathpar}
      \BiLNLdrulerl{} \and
      \BiLNLdrulets{} \and
      \BiLNLdrulemL{} \and
      \BiLNLdrulemR{} \and            
      \BiLNLdrulewkL{} \and
      \BiLNLdrulewkR{} \and
      \BiLNLdrulectrL{} \and
      \BiLNLdrulectrR{} \and
      \BiLNLdruleexL{} \and
      \BiLNLdruleexR{} \and
      \BiLNLdruleid{} \and      
      \BiLNLdrulecut{} \and
      \BiLNLdruleIL{} \and
      \BiLNLdruleIR{} \and
      \BiLNLdruleflL{} \and
      \BiLNLdruleflR{} \and
      \BiLNLdrulecL{} \and
      \BiLNLdrulecR{} \and
      \BiLNLdruledL{} \and
      \BiLNLdruledR{} \and
      \BiLNLdruleImpR{} \and
      \BiLNLdruleImpL{} \and
      \BiLNLdrulesL{} \and
      \BiLNLdrulesR{} 
    \end{mathpar}
  \end{mdframed}
  \caption{Inference Rules for L}
  \label{fig:ifr-L}
\end{figure}
% section embedding_l_in_bilnl_logic (end)

\section{BiLNL Term Assignment}
\label{sec:bilnl_term_assignment}
TODO
% section bilnl_term_assignment (end)


\section{Related Work}
\label{sec:related_work}
TODO
% section related_work (end)


\section{Conclusion}
\label{sec:conclusion}
TODO
% section conclusion (end)


\bibliographystyle{plainurl} \bibliography{ref}

\end{document}

%%% Local Variables: 
%%% mode: latex
%%% TeX-master: t
%%% End: 

