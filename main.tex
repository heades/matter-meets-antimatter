\documentclass{lmcs}
\usepackage{amssymb,amsmath}
\usepackage{cmll}
\usepackage{txfonts}
\usepackage{graphicx}
\usepackage{stmaryrd}
\usepackage{todonotes}
\usepackage{mathpartir}
\usepackage{hyperref}
\usepackage{mdframed}
\usepackage[barr]{xy}

\newtheorem{theorem}{Theorem}
\newtheorem{lemma}[theorem]{Lemma}
\newtheorem{corollary}[theorem]{Corollary}
\newtheorem{definition}[theorem]{Definition}
\newtheorem{proposition}[theorem]{Proposition}
\newtheorem{example}[theorem]{Example}

%% This renames Barr's \to to \mto.  This allows us to use \to for imp
%% and \mto for a inline morphism.
\let\mto\to
\let\to\relax
\newcommand{\to}{\rightarrow}
\newcommand{\ndto}[1]{\to_{#1}}
\newcommand{\ndwedge}[1]{\wedge_{#1}}

% Commands that are useful for writing about type theory and programming language design.
%% \newcommand{\case}[4]{\text{case}\ #1\ \text{of}\ #2\text{.}#3\text{,}#2\text{.}#4}
\newcommand{\interp}[1]{\llbracket #1 \rrbracket}
\newcommand{\normto}[0]{\rightsquigarrow^{!}}
\newcommand{\join}[0]{\downarrow}
\newcommand{\redto}[0]{\rightsquigarrow}
\newcommand{\nat}[0]{\mathbb{N}}
\newcommand{\fun}[2]{\lambda #1.#2}
\newcommand{\CRI}[0]{\text{CR-Norm}}
\newcommand{\CRII}[0]{\text{CR-Pres}}
\newcommand{\CRIII}[0]{\text{CR-Prog}}
\newcommand{\subexp}[0]{\sqsubseteq}
%% Must include \usepackage{mathrsfs} for this to work.

\date{}

\let\b\relax
\let\d\relax
\let\t\relax
\let\r\relax
\let\wn\relax

% Cat commands.
\newcommand{\powerset}[1]{\mathcal{P}(#1)}
\newcommand{\cat}[1]{\mathcal{#1}}
\newcommand{\func}[1]{\mathsf{#1}}
\newcommand{\catop}[1]{\cat{#1}^{\mathsf{op}}}
\newcommand{\Hom}[3]{\mathsf{Hom}_{\cat{#1}}(#2,#3)}
\newcommand{\limp}[0]{\multimap}
\newcommand{\colimp}[0]{\multimapdotinv}
\newcommand{\dial}[1]{\mathsf{Dial_{#1}}(\mathsf{Sets^{op}})}
\newcommand{\dialSets}[1]{\mathsf{Dial_{#1}}(\mathsf{Sets})}
\newcommand{\dcSets}[1]{\mathsf{DC_{#1}}(\mathsf{Sets})}
\newcommand{\sets}[0]{\mathsf{Sets}}
\newcommand{\obj}[1]{\mathsf{Obj}(#1)}
\newcommand{\mor}[1]{\mathsf{Mor(#1)}}
\newcommand{\id}[0]{\mathsf{id}}
\newcommand{\lett}[0]{\mathsf{let}\,}
\newcommand{\inn}[0]{\,\mathsf{in}\,}
\newcommand{\cur}[1]{\mathsf{cur}(#1)}
\newcommand{\curi}[1]{\mathsf{cur}^{-1}(#1)}
\newcommand{\m}[1]{\mathsf{m}_{#1}}
\newcommand{\n}[1]{\mathsf{n}_{#1}}
\newcommand{\b}[1]{\mathsf{b}_{#1}}
\newcommand{\d}[1]{\mathsf{d}_{#1}}
\newcommand{\p}[1]{\mathsf{p}_{#1}}
\newcommand{\q}[1]{\mathsf{q}_{#1}}
\newcommand{\t}[0]{\mathsf{t}}
\newcommand{\r}[1]{\mathsf{r}_{#1}}
\newcommand{\wn}[0]{\mathop{?}}

\newenvironment{changemargin}[2]{%
  \begin{list}{}{%
    \setlength{\topsep}{0pt}%
    \setlength{\leftmargin}{#1}%
    \setlength{\rightmargin}{#2}%
    \setlength{\listparindent}{\parindent}%
    \setlength{\itemindent}{\parindent}%
    \setlength{\parsep}{\parskip}%
  }%
  \item[]}{\end{list}}

%% Ott
\input{BiLNL-inc}

\urldef{\mailsa}\path|{heades}@augusta.edu|

\begin{document}

\title{Comonadic Matter Meets Monadic Anti-Matter: An Adjoint Model of Bi-Intuitionistic Logic}
\author{Harley Eades III}
\email{heades@augusta.edu}
\address{Computer and Information Sciences, Augusta University, Augusta, GA}

\maketitle 

\begin{abstract}

  Bi-intuitionistic logic (BINT) is a conservative extension of
  intuitionistic logic with perfect duality.  That is, BINT contains
  the usual intuitionistic logical connectives such as true,
  conjunction, and implication, but also their duals false,
  disjunction, and coimplication. One leading question with respect
  to BINT is, what does BINT look like across the three arcs -- logic,
  typed $\lambda$-calculi, and category theory -- of the
  Curry-Howard-Lambek correspondence?  A non-trivial (does not
  degenerate to a poset) categorical model of BINT is currently an
  open problem.  It is this open problem that this paper contributes
  to by providing the first fully developed categorical model of BINT.
  It is well-known that the linear counterpart, linear BINT, of BINT
  can be modeled in a symmetric monoidal closed category equipped with
  an additional monoidal structure that models par and a specified
  left adjoint to par called linear coimplication.  We call this
  model a symmetric bi-monoidal bi-closed category. In addition, it is
  well-known that intuitionistic logic has a categorical model of
  cartesian closed categories, and their dual cocartesian coclosed
  categories model cointuitionistic logic.  In this paper we exploit
  Benton's beautiful LNL models of linear logic to show that these
  three models can be mixed by requiring a symmetric monoidal
  adjunction between a cartesian closed category and the symmetric
  bi-monoidal bi-closed category, in addition to a symmetric monoidal
  adjunction between a cocartesian coclosed category and the
  symmetric bi-monoidal bi-closed category.  As a result of this
  mixture we obtain two modalities the usual comonadic of-course
  modality of linear logic, but also a monadic modality allowing for
  the embedding of cointuitionistic logic inside of linear BINT.
  Finally, using these modalities we show that BINT intuitionistic
  logic can be soundly modeled in this new categorical model. As a
  bi-product of this model we define BiLNL logic which can be seen as
  the mixture of intuitionistic logic with cointuitionistic logic
  inside of linear BINT.

\end{abstract}

\section{Introduction}
\label{sec:introduction}
TODO \cite{?}
% section introduction (end)

\section{Mixed Linear/Non-Linear Models of Bi-intuitionistic Logic: The Categorical Model}
\label{sec:the_categorical_model}

In this section we embark on the definition of a categorical model of
bi-intuitionistic logic via a categorical model of bi-intuitionistic
linear logic.  First, we summarize our main results.  Suppose
$(\cat{I}, 1, \times, \to)$ is a cartesian closed category, and
$(\cat{L}, \top, \otimes, \limp)$ is a symmetric monoidal closed
category.  Then relate these two categories with a symmetric monoidal
adjunction $\cat{I} : \func{F} \dashv \func{G} : \cat{L}$
(Definition~\ref{?}), where $\func{F}$ and $\func{G}$ are symmetric
monoidal functors.  The later point implies that there are natural
transformations $\m{X,Y} : \func{F}X \otimes \func{F}Y \mto
\func{F}(X \times Y)$ and $\n{A,B} : \func{G}A \times \func{G}B \mto
\func{G}(A \otimes B)$, and maps $\m\top : \top \mto \func{F}1$ and
$\n1 : 1 \mto \func{G}\top$ subject to several coherence conditions;
see Definition~\ref{?}.  Furthermore, the functor $\func{F}$ is strong
which means that $\m{X,Y}$ and $\m{\top}$ are isomorphisms.  This
setup turns out to be one of the most beautiful models of
intuitionistic linear logic called an LNL model due to Benton
\cite{Benton:1994}.  In fact, the linear modality of-course can be
defined by $!A = \func{F}(\func{G}(A))$ which defines a symmetric
monoidal comonad using the adjunction; see Section~2.2 of
\cite{Benton:1994}.  This model is much simpler than other known
models, and resulted in a logic called LNL logic which supports mixing
intuitionistic logic with linear logic.

Taking the dual of the previous model results in what we call dual LNL
models. They consist of a cocartesian coclosed category, $(\cat{C},
0, +, -)$, a symmetric monoidal coclosed category, $(\cat{L}', \perp,
\oplus, \colimp)$, where $\colimp : \cat{L}' \times \cat{L}' \mto
\cat{L}'$ is left adjoint to parr, and a symmetric monoidal adjunction
$\cat{L'} : H \dashv \perp : \cat{C}$.  We will show that dual LNL models
are a simplification of dual linear categories as defined by Bellin
\cite{Bellin:2012} in much of the same way that LNL models are a
simplification of linear categories.  In fact, we will define Girard's
exponential why-not by $? A = J(H(A))$, and hence, is the monad
induced by the adjunction.

Now we combine the two previous models into one model of
bi-intuitionistic linear logic.  Extend $\cat{L}$ with a second
symmetric monoidal structure, $(\perp, \oplus, \colimp)$, such that
$\otimes$ distributes over $\oplus$ and $\colimp : \cat{L} \times
\cat{L} \mto \cat{L}$ is left adjoint to $\oplus$, resulting in the
category $(\cat{L}, \top, \otimes, \limp, \perp, \oplus, \colimp)$,
called a symmetric monoidal bi-closed category.  We extend LNL models
into a new model of bi-intuitionistic logic called a mixed
linear/non-linear bi-intuitionistic model or a BiLNL model.  It
consists of a cartesian closed category, $(\cat{I}, 1, \times, \to)$,
a symmetric monoidal bi-closed category, $(\cat{L}, \top,\otimes,
\limp, \perp, \oplus, \colimp)$, a cocartesian coclosed category,
$(\cat{C}, 0,+, -)$, and a pair of symmetric monoidal adjoint
functors:
$$\bfig
\morphism(0,0)|a|/{@{>}@/^1em/}/<500,0>[\cat{I}`\cat{L};\func{F}]
\morphism(0,0)|b|/{@{<-}@/_1em/}/<500,0>[\cat{I}`\cat{L};\mathsf{G}]
\morphism(500,0)|a|/{@{>}@/^1em/}/<500,0>[\cat{L}`\func{C};\func{H}]
\morphism(500,0)|b|/{@{<-}@/_1em/}/<500,0>[\cat{L}`\func{C};\func{J}]
\place(250,0)[\dashv] \place(750,0)[\vdash] \efig$$ It is well known
that $\cat{I}$ is a model of intuitionistic logic and $\cat{C}$ is a
model of cointuitionistic logic, and so, the adjoint situation $F
\dashv G$ can be seen as a translation of intuitionistic logic into
the intuitionistic fragment, $(\top, \otimes, \limp)$, of $\cat{L}$,
and the adjoint situation $H \dashv J$ can be seen as a translation of
cointuitionistic logic into the cointuitionistic fragment, $(\perp,
\oplus, \colimp)$, of $\cat{L}$.  This we will show turns out to be a
model of bi-intuitionistic linear logic with modalities, but also as
one of the first categorical models of bi-intuitionistic logic by
exploiting the modalities of linear logic to embedded
bi-intuitionistic logic inside of bi-intuitionistic linear logic.

The previous model also induces an adjunction:
$$\bfig
\morphism(0,0)|a|/{@{>}@/^1em/}/<500,0>[\cat{I}`\cat{C};\func{F};\func{H}]
\morphism(0,0)|b|/{@{<-}@/_1em/}/<500,0>[\cat{I}`\cat{C};\func{J};\func{G}]
\place(250,0)[\dashv]\efig$$
\todo[inline]{Does this give us the double-negation monad/comonad?}

\subsection{Symmetric Monoidal Categories}
\label{subsec:symmetric_monoidal_categories}
We now introduce the necessary definitions related to symmetric
monoidal categories that our model will depend on.  Most of these
definitions are equivalent to the ones given by Benton
\cite{Benton:1994}.  In this section we also introduce distributive
categories, the notion of cocloser, and finally, the definition of
bilinear categories.  The reader may wish to simply skim this section,
but refer back to it when they encounter a definition or result they
do not know.

\begin{definition}
  \label{def:monoidal-category}
  A \textbf{symmetric monoidal category (SMC)} is a category, $\cat{M}$,
  with the following data:
  \begin{itemize}
  \item An object $\top$ of $\cat{M}$,
  \item A bi-functor $\otimes : \cat{M} \times \cat{M} \mto \cat{M}$,
  \item The following natural isomorphisms:
    \[
    \begin{array}{lll}
      \lambda_A : \top \otimes A \mto A\\
      \rho_A : A \otimes \top \mto A\\      
      \alpha_{A,B,C} : (A \otimes B) \otimes C \mto A \otimes (B \otimes C)\\
    \end{array}
    \]
  \item A symmetry natural transformation:
    \[
    \beta_{A,B} : A \otimes B \mto B \otimes A
    \]
  \item Subject to the following coherence diagrams:
    \begin{mathpar}
      \bfig
      \vSquares|ammmmma|/->`->```->``<-/[
        ((A \otimes B) \otimes C) \otimes D`
        (A \otimes (B \otimes C)) \otimes D`
        (A \otimes B) \otimes (C \otimes D)``
        A \otimes (B \otimes (C \otimes D))`
        A \otimes ((B \otimes C) \otimes D);
        \alpha_{A,B,C} \otimes \id_D`
        \alpha_{A \otimes B,C,D}```
        \alpha_{A,B,C \otimes D}``
        \id_A \otimes \alpha_{B,C,D}]      
      
      \morphism(1206,1000)|m|<0,-1000>[
        (A \otimes (B \otimes C)) \otimes D`
        A \otimes ((B \otimes C) \otimes D);
        \alpha_{A,B \otimes C,D}]
      \efig
      \and
      \bfig
      \hSquares|aammmaa|/->`->`->``->`->`->/[
        (A \otimes B) \otimes C`
        A \otimes (B \otimes C)`
        (B \otimes C) \otimes A`
        (B \otimes A) \otimes C`
        B \otimes (A \otimes C)`
        B \otimes (C \otimes A);
        \alpha_{A,B,C}`
        \beta_{A,B \otimes C}`
        \beta_{A,B} \otimes \id_C``
        \alpha_{B,C,A}`
        \alpha_{B,A,C}`
        \id_B \otimes \beta_{A,C}]
      \efig      
    \end{mathpar}
    \begin{mathpar}
      \bfig
      \Vtriangle[
        (A \otimes \top) \otimes B`
        A \otimes (\top \otimes B)`
        A \otimes B;
        \alpha_{A,\top,B}`
        \rho_{A}`
        \lambda_{B}]
      \efig
      \and
      \bfig
      \btriangle[
        A \otimes B`
        B \otimes A`
        A \otimes B;
        \beta_{A,B}`
        \id_{A \otimes B}`
        \beta_{B,A}]
      \efig
      \and
      \bfig
      \Vtriangle[
        \top \otimes A`
        A \otimes \top`
        A;
        \beta_{\top,A}`
        \lambda_A`
        \rho_A]
      \efig
    \end{mathpar}    
  \end{itemize}
\end{definition}

Categorical modeling implication requires that the model be closed;
which can be seen as an internalization of the notion of a morphism.
\begin{definition}
  \label{def:SMCC}
  A \textbf{symmetric monoidal closed category (SMCC)} is a symmetric
  monoidal category, $(\cat{M},\top,\otimes)$, such that, for any object
  $B$ of $\cat{M}$, the functor $- \otimes B : \cat{M} \mto \cat{M}$
  has a specified right adjoint.  Hence, for any objects $A$ and $C$
  of $\cat{M}$ there is an object $B \limp C$ of $\cat{M}$ and a
  natural bijection:
  \[
  \Hom{\cat{M}}{A \otimes B}{C} \cong \Hom{\cat{M}}{A}{B \limp C}
  \]
  We call the functor $\limp : \cat{M} \times \cat{M} \mto \cat{M}$
  the internal hom of $\cat{M}$.
\end{definition}

Symmetric monoidal closed categories can be seen as a model of
intuitionistic linear logic with a tensor product and implication.
What happens when we take the dual?  First, we have the following
result:
\begin{lemma}[Dual of Symmetric Monoidal Categories]
  \label{lemma:dual_of_symmetric_monoidal_categories}
  If $(\cat{M},\top,\otimes)$ is a symmetric monoidal category, then
  $\catop{M}$ is also a symmetric monoidal category.
\end{lemma}
The previous result follows from the fact that the structures making
up symmetric monoidal categories are isomorphisms, and so naturally
taking their opposite will yield another symmetric monoidal category.
To emphasize when we are thinking about a symmetric monoidal category
in the opposite we use the notion $(\cat{M},\perp,\oplus)$ which gives
the suggestion of $\oplus$ corresponding to a disjunctive tensor
product which we call the \textit{cotensor} of $\cat{M}$. The next
definition describes when a symmetric monoidal category is coclosed.
\begin{definition}
  \label{def:SMCC}
  A \textbf{symmetric monoidal coclosed category (SMCCC)} is a symmetric
  monoidal category, $(\cat{M},\perp,\oplus)$, such that, for any object
  $B$ of $\cat{M}$, the functor $- \oplus B : \cat{M} \mto \cat{M}$
  has a specified left adjoint.  Hence, for any objects $A$ and $C$
  of $\cat{M}$ there is an object $B \colimp C$ of $\cat{M}$ and a
  natural bijection:
  \[
  \Hom{\cat{M}}{C}{A \oplus B} \cong \Hom{\cat{M}}{B \colimp C}{A}
  \]
  We call the functor $\colimp : \cat{M} \times \cat{M} \mto \cat{M}$
  the internal cohom of $\cat{M}$.
\end{definition}

We combine the previous definitions into a single category.  First, we
define the notion of a distributive category due to Cockett and Seely
\cite{Cockett:1997}.
\begin{definition}
  \label{def:dist-cat}
  We call a symmetric monoidal category, $(\cat{M}, \top, \otimes,
  \perp, \oplus)$, a \textbf{distributive category} if there are
  natural transformations:
  \[
  \begin{array}{lll}
    \delta^L_{A,B,C} : A \otimes (B \oplus C) \mto (A \otimes B) \oplus C\\
    \delta^R_{A,B,C} : (B \oplus C) \otimes A \mto B \oplus (C \otimes A)
  \end{array}
  \]
  subject to several coherence diagrams.  Due to the large number of
  coherence diagrams we do not list them here, but they all can be
  found in Cockett and Seely's paper \cite{Cockett:1997}.
\end{definition}
\noindent
Requiring that the tensor and cotensor products have the corresponding
right and left adjoints results in the following definition.
\begin{definition}
  \label{def:bilinear-cat}
  A \textbf{bilinear category} is a distributive category $(\cat{M},
  \top, \otimes, \perp, \oplus)$ such that $(\cat{M}, \top, \otimes)$
  is closed, and $(\cat{M}, \perp, \oplus)$ is coclosed.  We will
  denote bi-linear categories by $(\cat{M}, \top, \otimes, \limp, \perp,
  \oplus, \colimp)$.
\end{definition}
Originally, Lambek defined bilinear categories to be similar to the
previous definition, but the tensor and cotensor were non-commutative
\cite{Cockett:1997a}, however, the bilinear categories given here
are. We retain the name in homage to his original work.  As we will
see below bilinear categories form the core of the work given here,
and are of crucial importance.

A symmetric monoidal category is a category with additional structure
subject to several coherence diagrams.  Thus, an ordinary functor is
not enough to capture this structure, and hence, the introduction of
symmetric monoidal functors.
\begin{definition}
  \label{def:SMCFUN}
  Suppose we are given two symmetric monoidal closed categories\\
  $(\cat{M}_1,\top_1,\otimes_1,\alpha_1,\lambda_1,\rho_1,\beta_1)$ and
  $(\cat{M}_2,\top_2,\otimes_2,\alpha_2,\lambda_2,\rho_2,\beta_2)$.  Then a
  \textbf{symmetric monoidal functor} is a functor $F : \cat{M}_1 \mto
  \cat{M}_2$, a map $m_\top : \top_2 \mto F\top_1$ and a natural transformation
  $m_{A,B} : FA \otimes_2 FB \mto F(A \otimes_1 B)$ subject to the
  following coherence conditions:
  \begin{mathpar}
    \bfig
    \vSquares|ammmmma|/->`->`->``->`->`->/[
      (FA \otimes_2 FB) \otimes_2 FC`
      FA \otimes_2 (FB \otimes_2 FC)`
      F(A \otimes_1 B) \otimes_2 FC`
      FA \otimes_2 F(B \otimes_1 C)`
      F((A \otimes_1 B) \otimes_1 C)`
      F(A \otimes_1 (B \otimes_1 C));
      {\alpha_2}_{FA,FB,FC}`
      m_{A,B} \otimes \id_{FC}`
      \id_{FA} \otimes m_{B,C}``
      m_{A \otimes_1 B,C}`
      m_{A,B \otimes_1 C}`
      F{\alpha_1}_{A,B,C}]
    \efig
    \end{mathpar}
%    \and
\begin{mathpar}
    \bfig
    \square|amma|/->`->`<-`->/<1000,500>[
      \top_2 \otimes_2 FA`
      FA`
      F\top_1 \otimes_2 FA`
      F(\top_1 \otimes_1 A);
      {\lambda_2}_{FA}`
      m_{\top} \otimes \id_{FA}`
      F{\lambda_1}_{A}`
      m_{\top_1,A}]
    \efig
    \and
    \bfig
    \square|amma|/->`->`<-`->/<1000,500>[
      FA \otimes_2 \top_2`
      FA`
      FA \otimes_2 F\top_1`
      F(A \otimes_1 \top_1);
      {\rho_2}_{FA}`
      \id_{FA} \otimes m_{\top}`
      F{\rho_1}_{A}`
      m_{A,\top_1}]
    \efig
     \end{mathpar}
     
      \begin{mathpar}
    \bfig
    \square|amma|/->`->`->`->/<1000,500>[
      FA \otimes_2 FB`
      FB \otimes_2 FA`
      F(A \otimes_1 B)`
      F(B \otimes_1 A);
      {\beta_2}_{FA,FB}`
      m_{A,B}`
      m_{B,A}`
      F{\beta_1}_{A,B}]
    \efig
  \end{mathpar}
\end{definition}

Naturally, since functors are enhanced to handle the additional
structure found in a symmetric monoidal category we must also extend
natural transformations, and adjunctions.
\begin{definition}
  \label{def:SMCNAT}
  Suppose $(\cat{M}_1,\top_1,\otimes_1)$ and $(\cat{M}_2,\top_2,\otimes_2)$
  are SMCs, and $(F,m)$ and $(G,n)$ are a symmetric monoidal functors
  between $\cat{M}_1$ and $\cat{M}_2$.  Then a \textbf{symmetric
    monoidal natural transformation} is a natural transformation,
  $f : F \mto G$, subject to the following coherence diagrams:
  \begin{mathpar}
    \bfig
    \square<1000,500>[
      FA \otimes_2 FB`
      F(A \otimes_1 B)`
      GA \otimes_2 GB`
      G(A \otimes_1 B);
      m_{A,B}`
      f_A \otimes_2 f_B`
      f_{A \otimes_1 B}`
      n_{A,B}]
    \efig
    \and
    \bfig
    \Vtriangle/->`<-`<-/[
      F\top_1`
      G\top_1`
      \top_2;
      f_{\top_1}`
      m_{\top_1}`
      n_{\top_1}]
    \efig
  \end{mathpar}  
\end{definition}  
\begin{definition}
  \label{def:SMCADJ}
  Suppose $(\cat{M}_1,\top_1,\otimes_1)$ and $(\cat{M}_2,\top_2,\otimes_2)$
  are SMCs, and $(F,m)$ is a symmetric monoidal functor between
  $\cat{M}_1$ and $\cat{M}_2$ and $(G,n)$ is a symmetric monoidal
  functor between $\cat{M}_2$ and $\cat{M}_1$.  Then a
  \textbf{symmetric monoidal adjunction} is an ordinary adjunction
  $\cat{M}_1 : F \dashv G : \cat{M}_2$ such that the unit,
  $\varepsilon : A \to GFA$, and the counit, $\eta_A : FGA \to A$, are
  symmetric monoidal natural transformations.  Thus, the following
  diagrams must commute:
  \begin{mathpar}
    \bfig
    \qtriangle|amm|<1000,500>[
      FGA \otimes_1 FGB`
      FG(A \otimes_1 B)`
      A \otimes_1 B;
      \q{A,B}`
      \eta_A \otimes_1 \eta_B`
      \eta_{A \otimes_1 B}]
    \efig
    \and
    \bfig
    \Vtriangle|amm|/->`<-`=/[
      FG\top_1`
      \top_1`
      \top_1;
      \eta_{\top_1}`
      \q{\top_1}`]
    \efig
    \and
    \bfig
    \dtriangle|mmb|<1000,500>[
      A \otimes_2 B`
      GFA \otimes_2 GFB`
      GF(A \otimes_2 B);
      \varepsilon_A \otimes_2 \varepsilon_B`
      \varepsilon_{A \otimes_2 B}`
      \p{A,B}]
    \efig
    \and
    \bfig
    \Vtriangle|amm|/->`=`<-/[
      \top_2`
      GF\top_2`
      \top_2;
      \varepsilon_{\top_2}``
      p_{\top_2}]
    \efig
  \end{mathpar}
  Note that $\p{}$ and $\q{}$ exist because $(FG,\q{})$ and
  $(GF,\p{})$ are symmetric monoidal functors.
\end{definition}
We will be defining, and making use of the of-course and why-not
exponentials from linear logic, but these correspond to a symmetric
monoidal comonad and a symmetric monoidal monad respectively, and so
we define these concepts next.  In addition, whenever we have a
symmetric monoidal adjunction, we immediately obtain a symmetric
monoidal comonad on the left, and a symmetric monoidal monad on the
right.
\begin{definition}
  \label{def:symm-monoidal-monad}
  A \textbf{symmetric monoidal monad} on a symmetric monoidal
  category $\cat{C}$ is a triple $(T,\eta, \mu)$, where
  $(T,\n{})$ is a symmetric monoidal endofunctor on $\cat{C}$,
  $\eta_A : A \mto TA$ and $\mu_A : T^2A \to TA$ are
  symmetric monoidal natural transformations, which make the following
  diagrams commute:
  \begin{mathpar}
    \bfig
    \square|ammb|<600,600>[
      T^3 A`
      T^2A`
      T^2A`
      TA;
      \mu_{TA}`
      T\mu_A`
      \mu_A`
      \mu_A]
    \efig
    \and
    \bfig
    \Atrianglepair/=`<-`=`->`<-/<600,600>[
      TA`
      TA`
      T^2 A`
      TA;`
      \mu_A``
      \eta_{TA}`
      T\eta_A]
    \efig
  \end{mathpar}
  The assumption that $\eta$ and $\mu$ are symmetric
  monoidal natural transformations amount to the following diagrams
  commuting:
  \begin{mathpar}
    \bfig
    \dtriangle|mmb|<1000,600>[
      A \otimes B`
      TA \otimes TB`
      T(A \otimes B);
      \eta_A \otimes \eta_B`
      \eta_A`
      \n{A,B}]    
    \efig
    \and
    \bfig
    \Vtriangle/->`=`<-/<600,600>[
      \top`
      T\top`
      \top;
      \eta_\top``
      \n{\top}]
    \efig
  \end{mathpar}
  \begin{mathpar}
    \bfig
    \square|ammm|/->`->``/<1050,600>[
      T^2 A \otimes T^2 B`
      T(TA \otimes TB)`
      TA \otimes TB`;
      \n{TA,TB}`
      \mu_A \otimes \mu_B``]

    \square(850,0)|ammm|/->``->`/<1050,600>[
      T(TA \otimes TB)`
      T^2(A \otimes B)``
      T(A \otimes B);
      T\n{A,B}``
      \mu_{A \otimes B}`]
    \morphism(-200,0)<2100,0>[TA \otimes TB`T(A \otimes B);\n{A,B}]
    \efig
    \and
    \bfig
    \square|ammb|/->`->`->`<-/<600,600>[
      \top`
      T\top`
      T\top`
      T^2\top;
      \n{\top}`
      \n{\top}`
      T\n{\top}`
      \mu_\top]
    \efig
  \end{mathpar}
\end{definition}
\noindent
Finally the dual concept, of a symmetric monoidal comonad.
\begin{definition}
  \label{def:symm-monoidal-comonad}
  A \textbf{symmetric monoidal comonad} on a symmetric monoidal
  category $\cat{C}$ is a triple $(T,\varepsilon, \delta)$, where
  $(T,\m{})$ is a symmetric monoidal endofunctor on $\cat{C}$,
  $\varepsilon_A : TA \mto A$ and $\delta_A : TA \to T^2 A$ are
  symmetric monoidal natural transformations, which make the following
  diagrams commute:
  \begin{mathpar}
    \bfig
    \square|amma|<600,600>[
      TA`
      T^2A`
      T^2A`
      T^3A;
      \delta_A`
      \delta_A`
      T\delta_A`
      \delta_{TA}]
    \efig
    \and
    \bfig
    \Atrianglepair/=`->`=`<-`->/<600,600>[
      TA`
      TA`
      T^2 A`
      TA;`
      \delta_A``
      \varepsilon_{TA}`
      T\varepsilon_A]
    \efig
  \end{mathpar}
  The assumption that $\varepsilon$ and $\delta$ are symmetric
  monoidal natural transformations amount to the following diagrams
  commuting:
  \begin{mathpar}
    \bfig
    \qtriangle|amm|/->`->`->/<1000,600>[
      TA \otimes TB`
      T(A \otimes B)`
      A \otimes B;
      \m{A,B}`
      \varepsilon_A \otimes \varepsilon_B`
    \varepsilon_{A \otimes B}]
    \efig
    \and
    \bfig
    \Vtriangle|amm|/->`<-`=/<600,600>[
      T\top`
      \top`
      \top;
      \m{\top}`
      \varepsilon_\top`]
    \efig    
  \end{mathpar}
  \begin{mathpar}
    \bfig
    \square|amab|/`->``->/<1050,600>[
      TA \otimes TB``
      T^2A \otimes T^2B`
      T(TA \otimes TB);`
      \delta_A \otimes \delta_B``
      \m{TA,TB}]
    \square(1050,0)|mmmb|/``->`->/<1050,600>[`
      T(A \otimes B)`
      T(TA \otimes TB)`
      T^2(A \otimes B);``
      \delta_{A \otimes B}`
      T\m{A,B}]
    \morphism(0,600)<2100,0>[TA \otimes TB`T(A \otimes B);\m{A,B}]
    \efig
    \and
    \bfig
    \square<600,600>[
      \top`
      T\top`
      T\top`
      T^2\top;
      \m{\top}`
      \m{\top}`
      \delta_\top`
      T\m{\top}]
    \efig
  \end{mathpar}
\end{definition}
% subsection symmetric_monoidal_categories (end)

\subsection{Cartesian Closed and Cocartesian Coclosed Categories}
\label{subsec:cartesian_closed_and_cocartesian_coclosed_categories}
The notion of a cartesian closed category is well-known, but for
completeness we define them here.  However, their dual is lesser
known, especially in computer science, and so we given their full
definition.  We also review some know results concerning cocartesian
coclosed categories and categories that are both cartesian closed and
cocartesian coclosed.
\begin{definition}
  \label{def:CC}
  A \textbf{cartesian category} is a category, $(\cat{C}, 1, \times)$,
  with an object, $1$, and a bi-functor, $\times : \cat{C} \times
  \cat{C} \mto \cat{C}$, such that for any object $A$ there is exactly
  one morphism $\diamond : A \to 1$, and for any morphisms $f : C \mto
  A$ and $g : C \mto B$ there is a morphism $\langle f , g \rangle : C
  \to A \times B$ subject to the following diagram:
  \[
  \bfig
  \Atrianglepair/->`->`->`<-`->/[C`A`A\times B`B;
    f`
    \langle f , g \rangle`
    g`
    \pi_1`
    \pi_2]
  \efig
  \]
\end{definition}
A cartesian category models conjunction by the product functor,
$\times : \cat{C} \times \cat{C} \mto \cat{C}$ , and the unit of
conjunction by the terminal object.  As we mention above modeling
implication requires closer, and since it is well-known that any
cartesian category is also a symmetric monoidal category the
definition of closer for a cartesian category is the same as the
definition of closer for a symmetric monoidal category
(Definition~\ref{def:SMCC}).  We denote the internal hom for cartesian
closed categories by $A \to B$.

The dual of a cartesian category is a cocartesian category.  They are
a model of intuitionistic logic with disjunction and its unit.
\begin{definition}
  \label{def:CC}
  A \textbf{cocartesian category} is a category, $(\cat{C}, 0, +)$,
  with an object, $0$, and a bi-functor,
  $+ : \cat{C} \times \cat{C} \mto \cat{C}$, such that for any object $A$ there is exactly
  one morphism $\Box : 0 \to A$, and for any morphisms $f : A \mto C$ and $g : B \mto C$
  there is a morphism $\lbrack f , g \rbrack : A + B \mto C$
  subject to the following diagram:
  \[
  \bfig
  \Atrianglepair/<-`<-`<-`->`<-/[C`A`A+B`B;
    f`
    \lbrack f, g \rbrack`
    g`
    \iota_1`
    \iota_2]
  \efig
  \]  
\end{definition}
Cocloser, just like closer for cartesian categories, is defined in
the same way that cocloser is defined for symmetric monoidal
categories, because cocartesian categories are also symmetric
monoidal categories.  Thus, a cocartesian category is coclosed if
there is a specified left-adjoint, which we denote $S - T$, to the
coproduct.

There are many examples of cocartesian coclosed categories.
Basically, any interesting cartesian category has an interesting dual,
and hence, induces an interesting cocartesian coclosed category.
The opposite of the category of sets and functions between them is
isomorphic to the category of complete atomic boolean algebras, and
both of which, are examples of cocartesian coclosed categories.  As
we mentioned above bi-linear categories \cite{Cockett:1997a} are
models of bi-linear logic where the left adjoint to the cotensor
models coimplication.  Similarly, cocartesian coclosed categories
model cointuitionistic logic with disjunction and intuitionistic
coimplication \cite{Crolard:2001,Bellin:2012}. \todo[inline]{Put more
  examples in here.}

We might now ask if we can a category can be both cartesian closed and
cocartesian coclosed just as bi-linear categories, but this turns out
to be where the matter meets antimatter in such away that the category
degenerates to a preorder.  That is, every homspace contains at most
one morphism.  We recall this proof here, which is due to Crolard
\cite{Crolard:2001}. We need a couple basic facts about cartesian
closed categories with initial objects.
\begin{lemma}
  \label{lemma:iso-prod-initial}
  In any cartesian category $\cat{C}$, if $0$ is an initial object in
  $\cat{C}$ and $\Hom{C}{A}{0}$ is non-empty, then $A \cong A \times 0$.
\end{lemma}
\begin{proof}
  This follows easily from the universial mapping property for products.
\end{proof}

\begin{lemma}
  \label{lemma:products-of-initial-gives-initial}
  In any cartesian closed category $C$, if $0$ is an initial
  object in $\cat{C}$, then so is $0 \times A$ for any object $A$
  of $\cat{C}$.
\end{lemma}
\begin{proof}
    We know that the universal morphism for the initial object is
    unique, and hence, the homspace $\Hom{C}{0}{A \Rightarrow B}$ for
    any object $B$ of $\cat{C}$ contains exactly one morphism.  Then
    using the right adjoint to the product functor we know that
    $\Hom{C}{0}{A \Rightarrow B} \cong \Hom{C}{0 \times A}{B}$, and
    hence, there is only one arrow between $0 \times A$ and $B$.
\end{proof}
\noindent
The following lemma is due to Joyal \cite{?}, and is key to the next
theorem.
\begin{lemma}[Joyal's]
  \label{lemma:joyals}
  In any cartesian closed category $\cat{C}$, if $0$ is an initial
  object in $\cat{C}$ and $\Hom{C}{A}{0}$ is non-empty, then $A$ is an
  initial object in $\cat{C}$.
\end{lemma}
\begin{proof}
  Suppose $\cat{C}$ is a cartesian closed category, such that, $0$ is
  an initial object in $\cat{C}$, and $A$ is an arbitrary object in
  $\cat{C}$.  Furthermore, suppose $\Hom{C}{A}{0}$ is non-empty.  By
  the first basic lemma above we know that $A \cong A \times 0$, and
  by the second $A \times 0$ is initial, thus $A$ is initial.
\end{proof}
Finally, the following theorem shows that any category that is both
cartesian closed and cocartesian coclosed is a preorder.
\begin{theorem}[(co)Cartesian (co)Closed Categories are Preorders]
  \label{thm:dengerate-to-preorder}
  If $\cat{C}$ is both cartesian closed and cocartesian coclosed, then
  for any two objects $A$ and $B$ of $\cat{C}$, $\Hom{C}{A}{B}$ has at
  most one element.
\end{theorem}
\begin{proof}
  Suppose $\cat{C}$ is both cartesian closed and cocartesian coclosed,
  and $A$ and $B$ are objects of $\cat{C}$.  Then by using the basic
  fact that the initial object is the unit to the coproduct, and the
  coproducts left adjoint we know the following:
  \[\Hom{C}{A}{B} \cong \Hom{C}{A}{0 + B} \cong \Hom{C}{B - A}{0}\]
  Therefore, by Joyal's theorem above $\Hom{C}{A}{B}$ has at most one
  element.
\end{proof}
\noindent
Notice that the previous result hinges on the fact that there are
initial and terminal objects, and thus, this result does not hold for
bi-linear categories, because the units to the tensor and cotensor are
not initial nor terminal.

The repercussions of this result are that if we do not want to work
with preorders, but do want to work with all of the structure, then we
must separate the two worlds.  Thus, this result can be seen as the
motivation for the current work.  We enforce the separation using
linear logic, but through the power of linear logic we show that
separation is not far.
% subsection cartesian_closed_and_cocartesian_coclosed_categories (end)

\subsection{A Mixed Linear/Non-Linear Model for Co-Intuitionistic Logic}
\label{subsec:a_mixed_linear/non-linear_model_for_co-intuitionistic_logic}

Benton \cite{Benton:1994} showed that from a LNL model it is possible
to construct a linear category, and vice versa.  Bellin
\cite{Bellin:2012} showed that the dual to linear categories are
sufficient to model co-intuitionistic linear logic. We show that from
the dual to a LNL model we can construct the dual to a linear
category, and vice versa, thus, carrying out the same program for
co-intuitionistic linear logic as Benton did for intuitionistic linear
logic.

Combining a symmetric monoidal coclosed category with a cocartesian
coclosed category via an symmetric monoidal adjunction defines a coLNL
model.
\begin{definition}
  \label{def:coLNL-model}
  A
  \textbf{mixed linear/non-linear model for co-intuitionistic logic (coLNL model)},
  $(\cat{L},\cat{C},\func{H},\func{J})$, consists of the following:
  \begin{itemize}
  \item[i.] a symmetric monoidal coclosed category $(\cat{L},\perp,\oplus,\colimp)$,
  \item[ii.] a cocartesian coclosed category $(\cat{C},0,+,-)$, and
  \item[iv.] a symmetric monoidal adjunction $\cat{L} : \func{H}
    \dashv \func{J} : \cat{C}$, where $\eta_A : A \mto
    \func{J}\,(\func{H}\,A)$ and $\varepsilon_R :
    \func{H}\,(\func{J}\,R) \mto R$ are the unit and counit of the adjunction
    respectively.
  \end{itemize}
\end{definition}
It is well-known that an adjunction $\cat{L} : \func{H} \dashv
\func{J} : \cat{C}$ induces a monad $\func{H};\func{J} : \cat{L} \mto
\cat{L}$, but when the adjunction is symmetric monoidal we obtain a
symmetric monoidal monad, in fact, $\func{H};\func{J}$ defines the
linear exponential why-not denoted $?A = \func{J}\,(\func{H}\,A))$.

\begin{lemma}[Symmetric Monoidal Monad]
  \label{lemma:symmetric_monoidal_monad}
  Given a coLNL model $(\cat{L},\cat{C},\func{H},\func{J})$, $H;J$
  defines a symmetric monoidal monad.
\end{lemma}
\begin{proof}
  All the diagrams defining a symmetric monoidal monad hold by the
  structure given by the adjunction.  For the complete proof see
  Appendix~\ref{subsec:proof_of_lemma:symmetric_monoidal_monad}.
\end{proof}

A $\wn-\text{algebra}$ is a pair $(\wn A, \mu_A)$, and is called
\textit{free} if it is an object of the full subcategory of all
$\wn-\text{algebra's}$ that is in adjunction with the category of sets
and functions such that the right adjoint is the forgetful functor.

% subsection a_mixed_linear/non-linear_model_for_co-intuitionistic_logic (end)

\subsection{A Mixed Bi-Linear/Non-Linear Model}
\label{subsec:a_mixed_bi-linear_non-linear_model}
\begin{definition}
  \label{def:biLNL-model}
  A \textbf{mixed bi-linear/non-linear model} consists of the
  following:
  \begin{itemize}
  \item[i.] a bi-linear category $(\cat{L},
    \top,\otimes,\limp,\perp,\oplus,\colimp)$,
  \item[ii.] a cartesian closed category $(\cat{I},1,\times,\to)$,
  \item[iii.] a cocartesian coclosed category $(\cat{C},0,+,-)$, and
  \item[iv.] two symmetric monoidal adjunctions $\cat{I} : F \dashv G
    : \cat{L}$ and $\cat{L} : H \dashv J : \cat{C}$.
  \end{itemize}
\end{definition}

% subsection a_mixed_bi-linear_non-linear_model (end)
% section the_categorical_model (end)

\section{Mixed Linear/Non-Linear Bi-intuitionistic Logic: BiLNL Logic}
\label{sec:bilnl_logic}

Following Benton's \cite{Benton:1994} lead we can define a mixed
linear/non-linear bi-intuitionistic logic, called BiLNL logic, based
on the categorical model given in the previous section.  BiLNL logic
consists of three fragments: an intuitionistic fragment, a
cointuitionistic fragment, and a linear bi-intuitionistic core
fragment.  This formalization allows us to have more control on the
mixture of the intuitionistic and cointuitionistic fragments so as to
allow for a proper categorical model.  Each of the fragments are
related through a syntactic formalization of the adjoint functors from
the BiLNL model.  First, we define the syntax of BiLNL logic, and then
discuss the inference rules for each fragment.
\begin{definition}
  \label{def:BiLNL-syntax}
  The syntax for BiLNL logic is defined as follows:
  \begin{center}
    \begin{math}
      \begin{array}{rlllllllllll}
        \text{(Worlds)} & W ::= [[w1]] \mid \cdots \mid [[wi]]\\
        \text{(Graphs)} & [[Gr]] ::= [[w1 <= w2]] \mid [[Gr1 , Gr2]]\\
        \text{(Intuitionistic Formulas)} & [[X]], [[Y]], [[Z]] ::= 1
          \mid [[X x Y]] \mid [[X -> Y]] \mid [[Gf A]]\\
        \text{(Cointuitionistic Formulas)} &  [[R]], [[S]], [[T]] ::= 0 \mid [[S + T]] \mid [[S - T]] \mid [[H A]]\\
        \text{(Linear Bi-intuitionistic Formulas)} &
             [[A]],[[B]],[[C]] ::= [[True]] \mid [[False]] \mid
             [[A (x) B]] \mid [[A (+) B]] \mid
             [[A -o B]] \mid [[A *- B]] \mid [[F X]] \mid [[J S]]\\
             \text{(Intuitionistic Contexts)}  & [[H]] ::= [[.]] \mid
                  [[X @ w]] \mid [[H1 , H2]]\\
        \text{(Cointuitionistic Contexts)}  & [[I]] ::= [[.]] \mid
             [[R @ w]] \mid [[I1 , I2]]\\
        \text{(Linear Bi-intuitionistic Contexts)}  &
             [[G]],[[D]] ::= [[.]] \mid [[A @ w]] \mid [[G1,G2]]\\        
      \end{array}
    \end{math}
  \end{center}
  Worlds may also be denoted by (potentially subscripted) $[[n]]$,
  $[[m]]$, and $[[o]]$.

  \ \\
  \noindent
  Sequents have the following syntax:
  \begin{center}
    \begin{math}
      \begin{array}{rll}
        \text{(Intuitionistic Sequents)}   & [[Gr;H |-I X@w]]\\
        \text{(Cointuitionistic Sequents)} & [[Gr;R@w |-C I]]\\
        \text{(LNL Bi-intuitionistic Sequents)} & [[Gr;H | G |-L D | I]]\\
      \end{array}
    \end{math}
  \end{center}
\end{definition}

The syntax of intuitionistic and cointuitionistic formulas are
typical.  I denote coimplication by $[[S - T]]$, but all the other
connectives are the usual ones. Linear bi-intuitionistic formulas are
denoted in somewhat of a non-traditional style. I denote the unit of
tensor by $[[True]]$ instead of the usual $[[1]]$, which is the unit
of intuitionistic conjunction, in addition, I denote par by $[[A (+)
    B]]$, instead of $[[A]] \parr [[B]]$.  Lastly, I denote linear
coimplication by $[[A *- B]]$ to emphasize its duality with linear
implication $[[A -o B]]$.  Each syntactic category of formulas
contains the respective functor from the BiLNL model, and thus, we
should view $F$ and $H$ as the left adjoints to $G$ and $J$
respectively.

Formulas in each type of context are annotated with a world, and each
sequent is annotated with a graph.  These graphs are syntactic
representations of Kripke models and are used to enforce intuitionism.
They were first used in bi-intuitionistic logic by Pinto and Uustalu
\cite{Pinto:2009} to enforce intuitionism in their logic L.  In fact,
we can see the linear core of BiLNL logic as the linear version of L.
The beauty of this type of formalization and the reason why this style
of logic was used by Pinto and Uustalu is that the logic L, and as
well as BiLNL logic, are complete for cut-free bi-intuitionistic
proofs.  This was a new result of Pinto and Uustalu, because earlier
formalizations of bi-intuitionistic logic \cite{Crolard:2001} used the
Dragalin restriction \cite{Dragalin:1988} to enforce intuitionism, but
this results in a failure of cut-elimination
\cite{Schellinx:1991,Pinto:2009}.

The expert reader will notice that it is not necessary to annotate the
sequents of intuitionistic and cointuitionistic logic with graphs and
worlds to enforce intuitionism and cointuitionism respectively. It is
well known that restricting the right and left contexts two a single
formula enforces intuitionism and cointuitionism respectively.
However, when mixing these two fragments with the linear core, which
requires the graphs to be intuitionistic, it is easier if they are
annotated.  If they were not, then a seemingly complex world inference
system would need to be designed to add the world constraints before
mixing with the linear core, and it is currently an open problem
whether this can be done.  Thus, with respect to intuitionistic and
cointuitionistic logic the graph and world annotations can be seen as
book keeping.

A second fact an expert reader will notice is that Kripke models are a
relational model of intuitionistic logic, but not intuitionistic
linear logic.  This is okay, because Kripke models enforce
intuitionism and not linearity, but it is well known how to enforce
linearity syntactically in the definition of the inference rules.  It
is also well-known that even in linear logic if sequents have multiple
hypothesis and multiple conclusions the logic becomes classical.
Thus, in BiLNL logic we combine both of these tools to enforce both
intuitionism and linearity.  We can simply view the graphs as an over
approximation of the relational constraints necessary to enforce both
intuitionism and linearity.  This also makes it easier to embed Pinto
and Uustalu's logic in BiLNL logic as we will do in
Section~\ref{sec:embedding_l_in_bilnl_logic}.

Sequents for the linear core have the form $[[Gr;H | G |-L D | I]]$.
Similarly to the sequents of Benton's LNL logic \cite{Benton:1994},
each context is separated for readability, but should actually be
understood as being able to be mixed, that is, the contexts $[[H]]$
and $[[G]]$ could be a single context, and so could $[[D]]$ and
$[[I]]$.  The sequent:
\[
[[Gr]];[[X1@w1]],\ldots,[[Xi@wi]] \mid [[A1@n1]],\ldots,[[Aj@nj]]
\vdash_{\mathsf{L}} [[B1@m1]],\ldots,[[Bk@mk]] \mid [[R1@o1]],\ldots,[[Rl@ol]]
\]
will be interpreted in a BiLNL model by a morphism of the following
form:
\[
[[F X1]] \otimes \cdots \otimes [[F Xi]] \otimes [[A1]] \otimes \cdots
\otimes [[Aj]] \mto^{f} [[B1]] \oplus \cdots \oplus [[Bk]] \oplus [[J R1]]
\oplus \cdots \oplus [[J Rl]]
\]
Thus, intuitionistic formulas in the linear core can be viewed as
being under the left adjoint $[[F]]$, and cointuitionistic formulas
as being under the right adjoint $[[J]]$.  This implies that even in
the model all types of formulas can be freely mixed.

As said above, BiLNL logic consists of three fragments: an
intuitionistic fragment, a cointuitionistic fragment, and a linear
bi-intuitionistic fragment.  Their relationships are captured by the
following diagram:
$$\bfig
\morphism(0,0)|a|/{@{>}@/^1em/}/<500,0>[\mathsf{INT}`\mathsf{BiL};\mathsf{F}]
\morphism(0,0)|b|/{@{<-}@/_1em/}/<500,0>[\mathsf{INT}`\mathsf{BiL};\mathsf{G}]
\morphism(500,0)|a|/{@{>}@/^1em/}/<500,0>[\mathsf{BiL}`\mathsf{coINT};\mathsf{H}]
\morphism(500,0)|b|/{@{<-}@/_1em/}/<500,0>[\mathsf{BiL}`\mathsf{coINT};\mathsf{J}]
\efig$$
\noindent
The inference rules for the intuitionistic fragment are defined in
Figure~\ref{fig:ifr-IL}, and the inference rules for the
cointuitionistic fragment can be found in Figure~\ref{fig:ifr-CL}.
Since the functor $[[Gf]]$ translates the linear fragment over to
intuitionistic logic, then the intuitionistic fragment must contain an
inference rule ($\BiLNLdrulename{I\_gR}$) for this functor, and the
same goes for the cointuitionistic fragment and the functor $[[Hf]]$
($\BiLNLdrulename{C\_hL}$).  Thus, these are the only interesting
rules of those two fragments. Each fragment contains rules for working
with the world constraints on sequents. For example, the rules
$\BiLNLdrulename{I\_rl}$ and $\BiLNLdrulename{I\_ts}$ make the
relation on worlds, $[[w1 <= w2]]$, a preorder, and the rules
$\BiLNLdrulename{I\_mL}$ and $\BiLNLdrulename{I\_mR}$ correspond to
monotonicity on formulas.  Similar rules exist for the other two
fragments as well.
\begin{figure}
  \begin{mdframed}
    \begin{mathpar}
      \BiLNLdruleIXXrl{} \and
      \BiLNLdruleIXXts{} \and
      \BiLNLdruleIXXmL{} \and
      \BiLNLdruleIXXmR{} \and      
      \BiLNLdruleIXXid{} \and
      \BiLNLdruleIXXcut{} \and
      \BiLNLdruleIXXwk{} \and
      \BiLNLdruleIXXcr{} \and
      \BiLNLdruleIXXex{} \and                  
      \BiLNLdruleIXXtL{} \and
      \BiLNLdruleIXXtR{} \and
      \BiLNLdruleIXXpL{} \and
      \BiLNLdruleIXXpR{} \and
      \BiLNLdruleIXXIL{} \and
      \BiLNLdruleIXXIR{} \and
      \BiLNLdruleIXXgR{}
    \end{mathpar}
  \end{mdframed}
  \caption{Inference Rules for BiLNL Logic: Intuitionistic Fragment}
  \label{fig:ifr-IL}
\end{figure}
\begin{figure}
  \begin{mdframed}
    \begin{mathpar}
      \BiLNLdruleCXXrl{} \and
      \BiLNLdruleCXXts{} \and
      \BiLNLdruleCXXmL{} \and
      \BiLNLdruleCXXmR{} \and      
      \BiLNLdruleCXXid{} \and
      \BiLNLdruleCXXcut{} \and
      \BiLNLdruleCXXwk{} \and
      \BiLNLdruleCXXcr{} \and
      \BiLNLdruleCXXex{} \and                  
      \BiLNLdruleCXXfL{} \and
      \BiLNLdruleCXXfR{} \and
      \BiLNLdruleCXXdL{} \and
      \BiLNLdruleCXXdR{} \and
      \BiLNLdruleCXXsL{} \and
      \BiLNLdruleCXXsR{} \and
      \BiLNLdruleCXXhL{}
    \end{mathpar}
  \end{mdframed}
  \caption{Inference Rules for BiLNL Logic: Cointuitionistic Fragment}
  \label{fig:ifr-CL}
\end{figure}

The linear core of BiLNL logic consists of a large number of rules,
and for this reason they have been broken up into multiple figures.
The inference rules for reflexivity, transitivity, and monotonicity
are defined in Figure~\ref{fig:ifr-biLNL-graph}.
\begin{figure}
  \begin{mdframed}
    \begin{mathpar}
      \BiLNLdruleLXXrl{} \and
      \BiLNLdruleLXXts{} \and
      \BiLNLdruleLXXmL{} \and
      \BiLNLdruleLXXmR{} \and
      \BiLNLdruleLXXImL{} \and
      \BiLNLdruleLXXCmR{}
    \end{mathpar}
  \end{mdframed}
  \caption{Inference Rules for BiLNL Logic: Abstract Kripke Graph Rules}
  \label{fig:ifr-biLNL-graph}
\end{figure}
Since the linear core is a mixed LNL logic there are monotonicity
rules for intuitionistic hypothesis, and cointuitionistic
conclusions.  The structural rules are defined in
Figure~\ref{fig:ifr-biLNL-structural}.
\begin{figure}
  \begin{mdframed}
    \begin{mathpar}
      \BiLNLdruleLXXwkL{} \and
      \BiLNLdruleLXXwkR{} \and
      \BiLNLdruleLXXctrL{} \and
      \BiLNLdruleLXXctrR{} \and
      \BiLNLdruleLXXexL{} \and
      \BiLNLdruleLXXexR{} \and
      \BiLNLdruleLXXIexL{} \and
      \BiLNLdruleLXXCexL{}
    \end{mathpar}
  \end{mdframed}
  \caption{Inference Rules for BiLNL Logic: Structural Rules}
  \label{fig:ifr-biLNL-structural}
\end{figure}
The most interesting rules are weakening and contraction for both
intuitionistic hypothesis and cointuitionistic conclusions.  The
identity and cut inference rules can be found in
Figure~\ref{fig:ifr-biLNL-id-cut}.
\begin{figure}
  \begin{mdframed}
    \begin{mathpar}
      \BiLNLdruleLXXid{} \and
      \BiLNLdruleLXXcut{} \and
      \BiLNLdruleLXXIcut{} \and
      \BiLNLdruleLXXCcut{} 
    \end{mathpar}
  \end{mdframed}
  \caption{Inference Rules for BiLNL Logic: Identity and Cut Rules}
  \label{fig:ifr-biLNL-id-cut}
\end{figure}
Similarly to Benton's LNL logic \cite{Benton:1994}, we also have cut
rules involving the different fragments:
\begin{mathpar}
  \small
  \BiLNLdruleLXXIcut{} \and \BiLNLdruleLXXCcut{} 
\end{mathpar}
The first rule is the same rule in LNL logic, but the second is new
and cuts a cointuitionistic conclusion in the linear core against the
hypothesis proven in the cointuitionistic fragment.

The rules for conjunction and tensor, and disjunction and par can be
found in Figure~\ref{fig:ifr-biLNL-conunction-tensor} and
Figure~\ref{fig:ifr-biLNL-disjunction-par} respectively.
\begin{figure}
  \begin{mdframed}
    \begin{mathpar}
      \BiLNLdruleLXXIL{} \and
      \BiLNLdruleLXXIR{} \and
      \BiLNLdruleLXXcL{} \and
      \BiLNLdruleLXXtL{} \and
      \BiLNLdruleLXXtR{} 
    \end{mathpar}
  \end{mdframed}
  \caption{Inference Rules for BiLNL Logic: Conjunction and Tensor Rules}
  \label{fig:ifr-biLNL-conunction-tensor}
\end{figure}
\begin{figure}
  \begin{mdframed}
    \begin{mathpar}
      \BiLNLdruleLXXflL{} \and
      \BiLNLdruleLXXflR{} \and
      \BiLNLdruleLXXdR{} \and
      \BiLNLdruleLXXpL{} \and
      \BiLNLdruleLXXpR{} 
    \end{mathpar}
  \end{mdframed}
  \caption{Inference Rules for BiLNL Logic: Disjunction and Par Rules}
  \label{fig:ifr-biLNL-disjunction-par}
\end{figure}
Just as we have said above, the new rules are the inference rules for
disjunction in the cointuitionistic context.  The inference rules for
intuitionistic implication and linear implication, and
cointuitionistic coimplication and linear coimplication are defined
in Figure~\ref{fig:ifr-biLNL-implication} and
Figure~\ref{fig:ifr-biLNL-coimplication}.  The most interesting rules
between these are the rules for coimplication.  
\begin{figure}
  \begin{mdframed}
    \begin{mathpar}
      \BiLNLdruleLXXImpL{} \and
      \BiLNLdruleLXXImpR{} \and
      \BiLNLdruleLXXIImpL{} 
    \end{mathpar}
  \end{mdframed}
  \caption{Inference Rules for BiLNL Logic: Implication Rules}
  \label{fig:ifr-biLNL-implication}
\end{figure}
\begin{figure}
  \begin{mdframed}
    \begin{mathpar}
      \BiLNLdruleLXXsL{} \and
      \BiLNLdruleLXXsR{} \and
      \BiLNLdruleLXXCsR{} 
    \end{mathpar}
  \end{mdframed}
  \caption{Inference Rules for BiLNL Logic: Coimplication Rules}
  \label{fig:ifr-biLNL-coimplication}
\end{figure}

Lastly, the inference rules for each of the functors are defined in
Figure~\ref{fig:ifr-biLNL-adjoint-functors}.  
\begin{figure}
  \begin{mdframed}
    \begin{mathpar}
      \BiLNLdruleLXXfL{} \and
      \BiLNLdruleLXXfR{} \and
      \BiLNLdruleLXXjL{} \and
      \BiLNLdruleLXXjR{} \and
      \BiLNLdruleLXXgL{} \and
      \BiLNLdruleLXXhR{} 
    \end{mathpar}
  \end{mdframed}
  \caption{Inference Rules for BiLNL Logic: Adjoint Functors Rules}
  \label{fig:ifr-biLNL-adjoint-functors}
\end{figure}
Using these rules it is possible to move between the fragments.  The
most restrictive rules are when coming from the intuitionistic or
cointuitionistic fragment into the linear core, or vice versa,
because the sequents must only contain data that is movable or
consistency could fail.
% section bilnl_logic (end)

\section{Embedding Bi-Intuitionistic Logic in BiLNL Logic}
\label{sec:embedding_l_in_bilnl_logic}

TODO
\begin{figure}
  \begin{mdframed}
    \begin{mathpar}
      \BiLNLdrulerl{} \and
      \BiLNLdrulets{} \and
      \BiLNLdrulemL{} \and
      \BiLNLdrulemR{} \and            
      \BiLNLdrulewkL{} \and
      \BiLNLdrulewkR{} \and
      \BiLNLdrulectrL{} \and
      \BiLNLdrulectrR{} \and
      \BiLNLdruleexL{} \and
      \BiLNLdruleexR{} \and
      \BiLNLdruleid{} \and      
      \BiLNLdrulecut{} \and
      \BiLNLdruleIL{} \and
      \BiLNLdruleIR{} \and
      \BiLNLdruleflL{} \and
      \BiLNLdruleflR{} \and
      \BiLNLdrulecL{} \and
      \BiLNLdrulecR{} \and
      \BiLNLdruledL{} \and
      \BiLNLdruledR{} \and
      \BiLNLdruleImpR{} \and
      \BiLNLdruleImpL{} \and
      \BiLNLdrulesL{} \and
      \BiLNLdrulesR{} 
    \end{mathpar}
  \end{mdframed}
  \caption{Inference Rules for L}
  \label{fig:ifr-L}
\end{figure}
% section embedding_l_in_bilnl_logic (end)

\section{BiLNL Term Assignment}
\label{sec:bilnl_term_assignment}
TODO
% section bilnl_term_assignment (end)


\section{Related Work}
\label{sec:related_work}
TODO
% section related_work (end)


\section{Conclusion}
\label{sec:conclusion}
TODO
% section conclusion (end)

\bibliographystyle{plainurl} \bibliography{ref}

\appendix
\section{Proofs}
\label{sec:proofs}

\subsection{Proof of Lemma~\ref{lemma:symmetric_comonoidal_isomorphisms}}
\label{subsec:proof_of_lemma:symmetric_comonoidal_isomorphisms}
We show that both of the  maps:
\[
  \begin{array}{lll}
    \jinv{R,S} := \func{J}R \oplus \func{J}S \mto^{\eta} \func{JH}(\func{J}R \oplus \func{J}S) \mto^{\func{J}\h{A,B}} \func{J}(\func{HJ}R + \func{HJ}S) \mto^{\J(\varepsilon_R + \varepsilon_S)} \func{J}(R + S)\\
  \\
  \jinv{\perp} := \perp \mto^{\eta} \func{JH}\perp \mto^{\func{J}\h{\perp}} \func{J}0
  \end{array}
  \]
are mutual inverses with $\j{R,S} : \func{J}(R + S) \mto \func{J}R
\oplus \func{J}S$ and $\j{\perp} : \perp \mto \func{J}0$ respectively.

\begin{itemize}
\item[Case.] The following diagram implies that $\jinv{R,S};\j{R,S} = \id$:
  \begin{diagram}
    \square|ammm|/->`->`->`<-/<950,500>[
      \func{J}R \oplus \func{J}S`
      \func{JH}(\func{J}R \oplus \func{J}S)`
      \func{JHJ}R \oplus \func{JHJ}S`
      \func{J}(\func{HJ}R + \func{HJ}S);
      \eta`
      \eta \oplus \eta`
      \func{J}\h{}`
      \j{}
    ]
    \dtriangle(-950,0)|amm|/=``<-/<950,500>[
      \func{J}R \oplus \func{J}S`
      \func{J}R \oplus \func{J}S`
      \func{JHJ}R \oplus \func{JHJ}S;``
      \func{J}\varepsilon \oplus \func{J}\varepsilon]

    \qtriangle(-950,-500)/`<-`->/<1900,500>[
      \func{J}R \oplus \func{J}S`
      \func{J}(\func{HJ}R + \func{HJ}S)`
      \func{J}(R + S);`
      \j{}`
      \func{J}(\varepsilon + \varepsilon)]        
  \end{diagram}
  The two top diagrams both commute because $\eta$ and $\varepsilon$
  are the unit and counit of the adjunction respectively, and the
  bottom diagram commutes by naturality of $\j{}$.
  
\item[Case.] The following diagram implies that $\j{R,S};\jinv{R,S} = \id$:
  \begin{diagram}
    \square|ammm|/->`->`->`->/<950,500>[
      \func{J}(R + S)`
      \func{J}R \oplus \func{J}S`
      \func{JHJ}(R + S)`
      \func{JH}(\func{J}R \oplus \func{J}S);
      \j{}`
      \eta`
      \eta`
      \func{JH}\j{}
    ]
    \dtriangle(-950,0)|amm|/=``<-/<950,500>[
      \func{J}(R + S)`
      \func{J}(R + S)`
      \func{JHJ}(R + S);``
      \func{J}\varepsilon]

    \qtriangle(-950,-500)/`<-`->/<1900,500>[
      \func{J}(R + S)`
      \func{JH}(\func{J}R \oplus \func{J}S)`
      \func{J}(\func{HJ}R + \func{HJ}S);`
      \func{J}(\varepsilon + \varepsilon)`
      \func{J}\h{}]
  \end{diagram}
  The top left and bottom diagrams both commute because $\eta$ and $\varepsilon$
  are the unit and counit of the adjunction respectively, and the
  top right diagram commutes by naturality of $\eta$.
  
\item[Case.] The following diagram implies that $\jinv{\perp};\j{\perp} = \id$:
  \begin{diagram}
    \square|amma|/->`=`->`<-/<950,500>[
      \perp`
      \func{JH}\perp`
      \perp`
      \func{J}0;
      \eta``
      \func{J}\h{\perp}`
      \j{\perp}]
  \end{diagram}
  This diagram holds because $\eta$ is the unit of the adjunction.

\item[Case.] The following diagram implies that $\j{\perp};\jinv{\perp} = \id$:        
  \begin{diagram}
    \Atriangle|aaa|/->`->`<-/<950,500>[
      \func{JHJ}0`
      \func{J}0`
      \func{JH}\perp;
      \func{J}\varepsilon`
      \func{JH}\j{\perp}`
      \func{J}\h{\perp}]

    \Dtriangle|aaa|/=`->`/<950,500>[
      \func{J}0`
      \func{JHJ}0`
      \func{J}0;`
      \eta`]

    \square/->``->`/<1900,1000>[
      \func{J}0`
      \perp`
      \func{J}0`
      \func{JH}\perp;
      \j{\perp}``
      \eta`]
  \end{diagram}
  The top-left and bottom diagrams commute because $\eta$ and
  $\varepsilon$ are the unit and counit of the adjunction
  respectively, and the top-right digram commutes by naturality of
  $\eta$.
\end{itemize}
% subsection proof_of_lemma~\ref{lemma:symmetric_comonoidal_isomorphisms} (end)

\subsection{Proof of Lemma~\ref{lemma:symmetric_comonoidal_monad}}
\label{subsec:proof_of_lemma:symmetric_comonoidal_monad}
We denote $H;J$ by $\wn$ throughout the proof, but expand this
definition where necessary.  First, it is easy to see that $?$ is
symmetric comonoidal, because it is the composition of two symmetric
monoidal functors.  Thus, the following diagrams all hold:
\begin{mathpar}
  \bfig
  \vSquares|ammmmma|/->`->`->``->`->`->/[
    \wn ((A \oplus B) \oplus C)`
    \wn (A \oplus B) \oplus \wn C`
    \wn (A \oplus (B \oplus C))`
    (\wn A \oplus \wn B) \oplus \wn C`
    \wn A \oplus \wn (B \oplus C))`
    \wn A \oplus (\wn B \oplus \wn C);
    m_{A \oplus B,C}`
    \wn \alpha_{A,B,C}`
    m_{A,B} \oplus \id_{\wn C}``
    m_{A,B \oplus C}`
    \alpha_{\wn A,\wn B,\wn C}`
    \id_{\wn A} \oplus m_{B,C}]    
  \efig
\end{mathpar}
%    \and
\begin{mathpar}
  \bfig
  \square|amma|/->`->`->`->/<1000,500>[
    \wn (\perp \oplus A)`
    \wn \perp \oplus \wn A`
    \wn A`
    \perp \oplus \wn A;
    m_{\perp,A}`
    \wn {\lambda}_{A}`
    m_{\perp} \oplus \id_{\wn A}`
    {\lambda^{-1}}_{\wn A}]
  \efig
  \and
  \bfig
  \square|amma|/->`->`->`->/<1000,500>[
    \wn (A \oplus \perp)`
    \wn A \oplus \wn \perp`
    \wn A`
    \wn A \oplus \perp;
    m_{A,\perp}`
    \wn {\rho}_{A}`
    \id_{\wn A} \oplus m_{\perp}`
       {\rho^{-1}}_{\wn A}]
  \efig
\end{mathpar}

\begin{diagram}
  \square|amma|/->`->`->`->/<1000,500>[
    \wn (A \oplus B)`
    \wn A \oplus \wn B`
    \wn (B \oplus A)`
    \wn B \oplus \wn A;
    m_{A,B}`
    \wn {\beta}_{A,B}`
        {\beta}_{\wn A,\wn B}`
        m_{B,A}]
\end{diagram}
Next we show that $(\wn,\eta,\mu)$ defines a monad where
$\eta_A : A \mto ?A$ is the unit of the adjunction, and
$\mu_A = \func{J}\varepsilon_{\func{H}\,A} : \wn\wn A \mto \wn A$.  It
suffices to show that every diagram of
Definition~\ref{def:symm-comonoidal-monad} holds.
\begin{itemize}
\item[Case.]
  $$\bfig
  \square|ammb|<600,600>[
    \wn^3 A`
    \wn^2 A`
    \wn^2 A`
    \wn A;
    \mu_{\wn A}`
    \wn\mu_A`
    \mu_A`
    \mu_A]
  \efig$$
  It suffices to show that the following diagram commutes:
  $$\bfig
  \square|ammb|<600,600>[
    \func{J}(\func{H}(\wn^2 A))`
    \func{J}(\func{H}\,\wn A)`
    \func{J}(\func{H}\,\wn A)`
    \func{J}(\func{H}\,A);
    \func{J}\varepsilon_{\func{H}\,\wn A}`
    \func{J}(\func{H}\,\mu_A)`
    \func{J}\varepsilon_{\func{H}\,A}`
    \func{J}\varepsilon_{\func{H}\,A}]
  \efig$$
  But this diagram is equivalent to the following:
  $$\bfig
  \square|ammb|<600,600>[
    \func{H}\func{JHJH} A`
    \func{H}\,\func{JH} A`
    \func{H}\,\func{JH} A`
    \func{H}\,A;
    \varepsilon_{\func{H}\,\func{JH} A}`
    \func{H}\,\func{J}\varepsilon_{\func{H}\,A}`
    \varepsilon_{\func{H}\,A}`
    \varepsilon_{\func{H}\,A}]
  \efig$$
  The previous diagram commutes by naturality of $\varepsilon$.

\item[Case.]
  $$\bfig
  \Atrianglepair/=`<-`=`->`<-/<600,600>[
    \wn A`
    \wn A`
    \wn^2 A`
    \wn A;`
    \mu_A``
    \eta_{\wn A}`
    \wn \eta_A]
  \efig$$
  It suffices to show that the following diagrams commutes:
  $$\bfig
  \Atrianglepair/=`<-`=`->`<-/<600,600>[
    JH A`
    JH A`
    JHJH A`
    JH A;`
    J\varepsilon_{HA}``
    \eta_{JH A}`
    JH \eta_A]
  \efig$$
  Both of these diagrams commute because $\eta$ and $\varepsilon$
  are the unit and counit of an adjunction.
\end{itemize}

It remains to be shown that $\eta$ and $\mu$ are both
symmetric comonoidal natural transformations, but this easily follows
from the fact that we know $\eta$ is by assumption, and that $\mu$
is because it is defined in terms of $\varepsilon$ which is a
symmetric comonoidal natural transformation.  Thus, all of the
following diagrams commute:
\begin{mathpar}
  \bfig
  \ptriangle|amm|/->`->`<-/<1000,600>[
    A \oplus B`
    \wn A \oplus \wn B`
    \wn (A \oplus B);
    \eta_A \oplus \eta_B`
    \eta_A`
    \n{A,B}]    
  \efig
  \and
  \bfig
  \Vtriangle/->`=`->/<600,600>[
    \perp`
    \wn\perp`
    \perp;
    \eta_\perp``
    \n{\perp}]
  \efig
\end{mathpar}
\begin{mathpar}
  \bfig
  \square|ammm|/->`->``/<1050,600>[
    \wn^2(A \oplus B)`
    \wn (\wn A \oplus \wn B)`
    \wn (A \oplus B)`;
    \wn\n{A,B}`
    \mu_{A \oplus B}``]

  \square(850,0)|ammm|/->``->`/<1050,600>[
    \wn (\wn A \oplus \wn B)`
    \wn^2 A \oplus \wn^2 B``
    \wn A \oplus \wn B;
    \n{\wn A,\wn B}``
    \mu_A \oplus \mu_B`]
  \morphism(-200,0)<2100,0>[\wn(A \oplus B)`\wn A \oplus \wn B;\n{A,B}]
  \efig
  \and
  \bfig
  \square|ammb|/->`->`->`->/<600,600>[
    \wn^2\perp`
    \wn\perp`
    \wn\perp`
    \perp;
    \wn\n{\perp}`
    \mu_\perp`
    \n{\perp}`
    \n{\perp}]
  \efig
\end{mathpar}
% subsection proof_of_lemma:symmetric_monoidal_monad (end)
% section proofs (end)

\end{document}

%%% Local Variables: 
%%% mode: latex
%%% TeX-master: t
%%% End: 

