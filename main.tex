\documentclass{lmcs}
\usepackage{amssymb,amsmath}
\usepackage{cmll}
\usepackage{txfonts}
\usepackage{graphicx}
\usepackage{stmaryrd}
\usepackage{todonotes}
\usepackage{mathpartir}
\usepackage{hyperref}
\usepackage{mdframed}
\usepackage[barr]{xy}

%% This renames Barr's \to to \mto.  This allows us to use \to for imp
%% and \mto for a inline morphism.
\let\mto\to
\let\to\relax
\newcommand{\to}{\rightarrow}
\newcommand{\ndto}[1]{\to_{#1}}
\newcommand{\ndwedge}[1]{\wedge_{#1}}

% Commands that are useful for writing about type theory and programming language design.
%% \newcommand{\case}[4]{\text{case}\ #1\ \text{of}\ #2\text{.}#3\text{,}#2\text{.}#4}
\newcommand{\interp}[1]{\llbracket #1 \rrbracket}
\newcommand{\normto}[0]{\rightsquigarrow^{!}}
\newcommand{\join}[0]{\downarrow}
\newcommand{\redto}[0]{\rightsquigarrow}
\newcommand{\nat}[0]{\mathbb{N}}
\newcommand{\fun}[2]{\lambda #1.#2}
\newcommand{\CRI}[0]{\text{CR-Norm}}
\newcommand{\CRII}[0]{\text{CR-Pres}}
\newcommand{\CRIII}[0]{\text{CR-Prog}}
\newcommand{\subexp}[0]{\sqsubseteq}
%% Must include \usepackage{mathrsfs} for this to work.

\date{}

% Cat commands.
\newcommand{\powerset}[1]{\mathcal{P}(#1)}
\newcommand{\cat}[1]{\mathcal{#1}}
\newcommand{\catop}[1]{\cat{#1}^{\mathsf{op}}}
\newcommand{\Hom}[3]{\mathsf{Hom}_{\cat{#1}}(#2,#3)}
\newcommand{\limp}[0]{\multimap}
\newcommand{\colimp}[0]{\multimapdotinv}
\newcommand{\dial}[1]{\mathsf{Dial_{#1}}(\mathsf{Sets^{op}})}
\newcommand{\dialSets}[1]{\mathsf{Dial_{#1}}(\mathsf{Sets})}
\newcommand{\dcSets}[1]{\mathsf{DC_{#1}}(\mathsf{Sets})}
\newcommand{\sets}[0]{\mathsf{Sets}}
\newcommand{\obj}[1]{\mathsf{Obj}(#1)}
\newcommand{\mor}[1]{\mathsf{Mor(#1)}}
\newcommand{\id}[0]{\mathsf{id}}
\newcommand{\lett}[0]{\mathsf{let}\,}
\newcommand{\inn}[0]{\,\mathsf{in}\,}
\newcommand{\cur}[1]{\mathsf{cur}(#1)}
\newcommand{\curi}[1]{\mathsf{cur}^{-1}(#1)}

\newenvironment{changemargin}[2]{%
  \begin{list}{}{%
    \setlength{\topsep}{0pt}%
    \setlength{\leftmargin}{#1}%
    \setlength{\rightmargin}{#2}%
    \setlength{\listparindent}{\parindent}%
    \setlength{\itemindent}{\parindent}%
    \setlength{\parsep}{\parskip}%
  }%
  \item[]}{\end{list}}

%% Ott
\input{BiLNL-inc}

\urldef{\mailsa}\path|{heades}@augusta.edu|

\begin{document}

\title{Comonadic Matter Meets Monadic Anti-Matter: An Adjoint Model of Bi-Intuitionistic Logic}
\author{Harley Eades III}
\email{heades@augusta.edu}
\address{Computer and Information Sciences, Augusta University, Augusta, GA}

\maketitle 

\begin{abstract}

  Bi-intuitionistic logic (BINT) is a conservative extension of
  intuitionistic logic with perfect duality.  That is, BINT contains
  the usual intuitionistic logical connectives such as true,
  conjunction, and implication, but also their duals false,
  disjunction, and co-implication. One leading question with respect
  to BINT is, what does BINT look like across the three arcs -- logic,
  typed $\lambda$-calculi, and category theory -- of the
  Curry-Howard-Lambek correspondence?  A non-trivial (does not
  degenerate to a poset) categorical model of BINT is currently an
  open problem.  It is this open problem that this paper contributes
  to by providing the first fully developed categorical model of BINT.
  It is well-known that the linear counterpart, linear BINT, of BINT
  can be modeled in a symmetric monoidal closed category equipped with
  an additional monoidal structure that models par and a specified
  left adjoint to par called linear co-implication.  We call this
  model a symmetric bi-monoidal bi-closed category. In addition, it is
  well-known that intuitionistic logic has a categorical model of
  cartesian closed categories, and their dual co-cartesian co-closed
  categories model co-intuitionistic logic.  In this paper we exploit
  Benton's beautiful LNL models of linear logic to show that these
  three models can be mixed by requiring a symmetric monoidal
  adjunction between a cartesian closed category and the symmetric
  bi-monoidal bi-closed category, in addition to a symmetric monoidal
  adjunction between a co-cartesian co-closed category and the
  symmetric bi-monoidal bi-closed category.  As a result of this
  mixture we obtain two modalities the usual comonadic of-course
  modality of linear logic, but also a monadic modality allowing for
  the embedding of co-intuitionistic logic inside of linear BINT.
  Finally, using these modalities we show that BINT intuitionistic
  logic can be soundly modeled in this new categorical model. As a
  bi-product of this model we define BiLNL logic which can be seen as
  the mixture of intuitionistic logic with co-intuitionistic logic
  inside of linear BINT.

\end{abstract}

\section{Introduction}
\label{sec:introduction}
TODO \cite{?}
% section introduction (end)

\section{Mixed Linear/Non-Linear Models of Bi-intuitionistic Logic: The Categorical Model}
\label{sec:the_categorical_model}
TODO
% section the_categorical_model (end)

\section{Mixed Linear/Non-Linear Bi-intuitionistic Logic: BiLNL Logic}
\label{sec:bilnl_logic}

\begin{figure}
  \begin{mdframed}
    \begin{mathpar}
      \BiLNLdruleIXXrl{} \and
      \BiLNLdruleIXXts{} \and
      \BiLNLdruleIXXid{} \and
      \BiLNLdruleIXXcut{} \and
      \BiLNLdruleIXXwk{} \and
      \BiLNLdruleIXXcr{} \and
      \BiLNLdruleIXXex{} \and                  
      \BiLNLdruleIXXmL{} \and
      \BiLNLdruleIXXmR{} \and
      \BiLNLdruleIXXtL{} \and
      \BiLNLdruleIXXtR{} \and
      \BiLNLdruleIXXpL{} \and
      \BiLNLdruleIXXpR{} \and
      \BiLNLdruleIXXIL{} \and
      \BiLNLdruleIXXIR{} \and
      \BiLNLdruleIXXgR{}
    \end{mathpar}
  \end{mdframed}
  \caption{Intuitionistic Fragment of L}
  \label{fig:ifr-IL}
\end{figure}

\begin{figure}
  \begin{mdframed}
    \begin{mathpar}
      \BiLNLdruleCXXrl{} \and
      \BiLNLdruleCXXts{} \and
      \BiLNLdruleCXXid{} \and
      \BiLNLdruleCXXcut{} \and
      \BiLNLdruleCXXwk{} \and
      \BiLNLdruleCXXcr{} \and
      \BiLNLdruleCXXex{} \and                  
      \BiLNLdruleCXXmL{} \and
      \BiLNLdruleCXXmR{} \and
      \BiLNLdruleCXXfL{} \and
      \BiLNLdruleCXXfR{} \and
      \BiLNLdruleCXXdL{} \and
      \BiLNLdruleCXXdR{} \and
      \BiLNLdruleCXXsL{} \and
      \BiLNLdruleCXXsR{} \and
      \BiLNLdruleCXXhL{}
    \end{mathpar}
  \end{mdframed}
  \caption{Co-intuitionistic Fragment of L}
  \label{fig:ifr-CL}
\end{figure}

\begin{figure}
  \begin{mdframed}
    \begin{mathpar}
      \BiLNLdruleLLXXrl{} \and
      \BiLNLdruleLLXXts{} \and
      \BiLNLdruleLLXXmL{} \and
      \BiLNLdruleLLXXmR{} \and
      \BiLNLdruleLLXXImL{} \and
      \BiLNLdruleLLXXCmR{}
    \end{mathpar}
  \end{mdframed}
  \caption{Inference Rules for BiLNL Logic: Abstract Kripke Graph Rules}
  \label{fig:ifr-biLNL-graph}
\end{figure}

\begin{figure}
  \begin{mdframed}
    \begin{mathpar}
      \BiLNLdruleLLXXwkL{} \and
      \BiLNLdruleLLXXwkR{} \and
      \BiLNLdruleLLXXctrL{} \and
      \BiLNLdruleLLXXctrR{} \and
      \BiLNLdruleLLXXexL{} \and
      \BiLNLdruleLLXXexR{} \and
      \BiLNLdruleLLXXIexL{} \and
      \BiLNLdruleLLXXCexL{}
    \end{mathpar}
  \end{mdframed}
  \caption{Inference Rules for BiLNL Logic: Structural Rules}
  \label{fig:ifr-biLNL-structural}
\end{figure}

\begin{figure}
  \begin{mdframed}
    \begin{mathpar}
      \BiLNLdruleLLXXid{} \and
      \BiLNLdruleLLXXcut{} \and
      \BiLNLdruleLLXXIcut{} \and
      \BiLNLdruleLLXXCcut{} 
    \end{mathpar}
  \end{mdframed}
  \caption{Inference Rules for BiLNL Logic: Identity and Cut Rules}
  \label{fig:ifr-biLNL-id-cut}
\end{figure}

\begin{figure}
  \begin{mdframed}
    \begin{mathpar}
      \BiLNLdruleLLXXIL{} \and
      \BiLNLdruleLLXXIR{} \and
      \BiLNLdruleLLXXcL{} \and
      \BiLNLdruleLLXXtL{} \and
      \BiLNLdruleLLXXtR{} 
    \end{mathpar}
  \end{mdframed}
  \caption{Inference Rules for BiLNL Logic: Conjunction and Tensor Rules}
  \label{fig:ifr-biLNL-conunction-tensor}
\end{figure}

\begin{figure}
  \begin{mdframed}
    \begin{mathpar}
      \BiLNLdruleLLXXflL{} \and
      \BiLNLdruleLLXXflR{} \and
      \BiLNLdruleLLXXdR{} \and
      \BiLNLdruleLLXXpL{} \and
      \BiLNLdruleLLXXpR{} 
    \end{mathpar}
  \end{mdframed}
  \caption{Inference Rules for BiLNL Logic: Disjunction and Par Rules}
  \label{fig:ifr-biLNL-disjunction-par}
\end{figure}

\begin{figure}
  \begin{mdframed}
    \begin{mathpar}
      \BiLNLdruleLLXXImpL{} \and
      \BiLNLdruleLLXXImpR{} \and
      \BiLNLdruleLLXXIImpL{} 
    \end{mathpar}
  \end{mdframed}
  \caption{Inference Rules for BiLNL Logic: Implication Rules}
  \label{fig:ifr-biLNL-implication}
\end{figure}

\begin{figure}
  \begin{mdframed}
    \begin{mathpar}
      \BiLNLdruleLLXXsL{} \and
      \BiLNLdruleLLXXsR{} \and
      \BiLNLdruleLLXXCsR{} 
    \end{mathpar}
  \end{mdframed}
  \caption{Inference Rules for BiLNL Logic: Co-implication Rules}
  \label{fig:ifr-biLNL-co-implication}
\end{figure}

\begin{figure}
  \begin{mdframed}
    \begin{mathpar}
      \BiLNLdruleLLXXfL{} \and
      \BiLNLdruleLLXXfR{} \and
      \BiLNLdruleLLXXjL{} \and
      \BiLNLdruleLLXXjR{} \and
      \BiLNLdruleLLXXgL{} \and
      \BiLNLdruleLLXXhR{} 
    \end{mathpar}
  \end{mdframed}
  \caption{Inference Rules for BiLNL Logic: Adjoint Functors Rules}
  \label{fig:ifr-biLNL-adjoint-functors}
\end{figure}
% section bilnl_logic (end)

\section{Embedding Bi-Intuitionistic Logic in BiLNL Logic}
\label{sec:embedding_l_in_bilnl_logic}

TODO
\begin{figure}
  \begin{mdframed}
    \begin{mathpar}
      \BiLNLdrulerl{} \and
      \BiLNLdrulets{} \and
      \BiLNLdrulecut{} \and
      \BiLNLdruleid{} \and
      \BiLNLdrulemL{} \and
      \BiLNLdrulemR{} \and
      \BiLNLdruletL{} \and
      \BiLNLdruletR{} \and
      \BiLNLdrulefL{} \and
      \BiLNLdrulefR{} \and
      \BiLNLdruleaL{} \and
      \BiLNLdruleaR{} \and
      \BiLNLdruledL{} \and
      \BiLNLdruledR{} \and
      \BiLNLdruleiL{} \and
      \BiLNLdruleiR{} \and
      \BiLNLdrulesL{} \and
      \BiLNLdrulesR{} 
    \end{mathpar}
  \end{mdframed}
  \caption{Inference Rules for L}
  \label{fig:ifr-L}
\end{figure}
% section embedding_l_in_bilnl_logic (end)

\section{BiLNL Term Assignment}
\label{sec:bilnl_term_assignment}
TODO
% section bilnl_term_assignment (end)


\section{Related Work}
\label{sec:related_work}
TODO
% section related_work (end)


\section{Conclusion}
\label{sec:conclusion}
TODO
% section conclusion (end)


\bibliographystyle{plainurl} \bibliography{ref}

\end{document}

%%% Local Variables: 
%%% mode: latex
%%% TeX-master: t
%%% End: 

