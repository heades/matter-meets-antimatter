\documentclass[letterpaper,USenglish]{lipics-v2016}

\usepackage{amssymb,amsmath}
\usepackage{cmll}
\usepackage{txfonts}
\usepackage{graphicx}
\usepackage{stmaryrd}
\usepackage{todonotes}
\usepackage{mathpartir}
\usepackage{hyperref}
\usepackage{mdframed}
\usepackage[barr]{xy}

%% This renames Barr's \to to \mto.  This allows us to use \to for imp
%% and \mto for a inline morphism.
\let\mto\to
\let\to\relax
\newcommand{\to}{\rightarrow}
\newcommand{\ndto}[1]{\to_{#1}}
\newcommand{\ndwedge}[1]{\wedge_{#1}}

% Commands that are useful for writing about type theory and programming language design.
%% \newcommand{\case}[4]{\text{case}\ #1\ \text{of}\ #2\text{.}#3\text{,}#2\text{.}#4}
\newcommand{\interp}[1]{\llbracket #1 \rrbracket}
\newcommand{\normto}[0]{\rightsquigarrow^{!}}
\newcommand{\join}[0]{\downarrow}
\newcommand{\redto}[0]{\rightsquigarrow}
\newcommand{\nat}[0]{\mathbb{N}}
\newcommand{\fun}[2]{\lambda #1.#2}
\newcommand{\CRI}[0]{\text{CR-Norm}}
\newcommand{\CRII}[0]{\text{CR-Pres}}
\newcommand{\CRIII}[0]{\text{CR-Prog}}
\newcommand{\subexp}[0]{\sqsubseteq}
%% Must include \usepackage{mathrsfs} for this to work.

\date{}

% Cat commands.
\newcommand{\powerset}[1]{\mathcal{P}(#1)}
\newcommand{\cat}[1]{\mathcal{#1}}
\newcommand{\catop}[1]{\cat{#1}^{\mathsf{op}}}
\newcommand{\Hom}[3]{\mathsf{Hom}_{\cat{#1}}(#2,#3)}
\newcommand{\limp}[0]{\multimap}
\newcommand{\colimp}[0]{\multimapdotinv}
\newcommand{\dial}[1]{\mathsf{Dial_{#1}}(\mathsf{Sets^{op}})}
\newcommand{\dialSets}[1]{\mathsf{Dial_{#1}}(\mathsf{Sets})}
\newcommand{\dcSets}[1]{\mathsf{DC_{#1}}(\mathsf{Sets})}
\newcommand{\sets}[0]{\mathsf{Sets}}
\newcommand{\obj}[1]{\mathsf{Obj}(#1)}
\newcommand{\mor}[1]{\mathsf{Mor(#1)}}
\newcommand{\id}[0]{\mathsf{id}}
\newcommand{\lett}[0]{\mathsf{let}\,}
\newcommand{\inn}[0]{\,\mathsf{in}\,}
\newcommand{\cur}[1]{\mathsf{cur}(#1)}
\newcommand{\curi}[1]{\mathsf{cur}^{-1}(#1)}

\newenvironment{changemargin}[2]{%
  \begin{list}{}{%
    \setlength{\topsep}{0pt}%
    \setlength{\leftmargin}{#1}%
    \setlength{\rightmargin}{#2}%
    \setlength{\listparindent}{\parindent}%
    \setlength{\itemindent}{\parindent}%
    \setlength{\parsep}{\parskip}%
  }%
  \item[]}{\end{list}}

%% Large Dot for nu:
% A larger \cdot, use with \Cdot[s] where s is a scaling factor like 1 or 2
%
% Code from: http://tex.stackexchange.com/questions/81729/a-medium-dot-in-math-mode

\newcommand*{\Cdot}[1][1.25]{%
  \mathpalette{\CdotAux{#1}}\cdot%
}
\newdimen\CdotAxis
\newcommand*{\CdotAux}[3]{%
  {%
    \settoheight\CdotAxis{$#2\vcenter{}$}%
    \sbox0{%
      \raisebox\CdotAxis{%
        \scalebox{#1}{%
          \raisebox{-\CdotAxis}{%
            $\mathsurround=0pt #2#3$%
          }%
        }%
      }%
    }%
    % Remove depth that arises from scaling.
    \dp0=0pt %
    % Decrease scaled height.
    \sbox2{$#2\bullet$}%
    \ifdim\ht2<\ht0 %
      \ht0=\ht2 %
    \fi
    % Use the same width as the original \cdot.
    \sbox2{$\mathsurround=0pt #2#3$}%
    \hbox to \wd2{\hss\usebox{0}\hss}%
  }%
}

%% Ott
\input{L-inc}

\title{Comonadic Matter Meets Monadic Anti-Matter: An Adjoint Model of Bi-Intuitionistic Logic}
\titlerunning{Comonadic Matter Meets Monadic Anti-Matter}

\author{Harley Eades III}
\affil{Computer and Information Sciences, Augusta University,
  Augusta, GA, \texttt{heades@augusta.edu}}
\authorrunning{H. Eades III}
\Copyright{Harley D. Eades III}

\subjclass{Dummy classification -- please refer to
  \url{http://www.acm.org/about/class/ccs98-html}}
\keywords{Dummy keyword -- please provide 1--5 keywords}

\begin{document}

\maketitle 

\begin{abstract}

\end{abstract}

\section{Introduction}
\label{sec:introduction}
TODO \cite{?}
% section introduction (end)

\begin{figure}
  \begin{mdframed}
    \begin{mathpar}
      \LdruleIXXrl{} \and
      \LdruleIXXts{} \and
      \LdruleIXXid{} \and
      \LdruleIXXcut{} \and
      \LdruleIXXwk{} \and
      \LdruleIXXcr{} \and
      \LdruleCXXex{} \and                  
      \LdruleIXXmL{} \and
      \LdruleIXXmR{} \and
      \LdruleIXXtL{} \and
      \LdruleIXXtR{} \and
      \LdruleIXXaL{} \and
      \LdruleIXXaR{} \and
      \LdruleIXXiL{} \and
      \LdruleIXXiR{} \and
      \LdruleIXXgR{}
    \end{mathpar}
  \end{mdframed}
  \caption{Intuitionistic Fragment of L}
  \label{fig:ifr-IL}
\end{figure}

\begin{figure}
  \begin{mdframed}
    \begin{mathpar}
      \LdruleCXXrl{} \and
      \LdruleCXXts{} \and
      \LdruleCXXid{} \and
      \LdruleCXXcut{} \and
      \LdruleCXXwk{} \and
      \LdruleCXXcr{} \and
      \LdruleCXXex{} \and                  
      \LdruleCXXmL{} \and
      \LdruleCXXmR{} \and
      \LdruleCXXfL{} \and
      \LdruleCXXfR{} \and
      \LdruleCXXdL{} \and
      \LdruleCXXdR{} \and
      \LdruleCXXsL{} \and
      \LdruleCXXsR{} \and
      \LdruleCXXhL{}
    \end{mathpar}
  \end{mdframed}
  \caption{Co-intuitionistic Fragment of L}
  \label{fig:ifr-CL}
\end{figure}

\begin{figure}
  \begin{mdframed}
    \begin{mathpar}
      \LdruleLLXXrl{} \and
      \LdruleLLXXts{} \and
      \LdruleLLXXmL{} \and
      \LdruleLLXXmR{} \and
      \LdruleLLXXImL{} \and
      \LdruleLLXXCmR{}
    \end{mathpar}
  \end{mdframed}
  \caption{Inference Rules for BiLNL Logic: Abstract Kripke Graph Rules}
  \label{fig:ifr-CL}
\end{figure}

\begin{figure}
  \begin{mdframed}
    \begin{mathpar}
      \LdruleLLXXwkL{} \and
      \LdruleLLXXwkR{} \and
      \LdruleLLXXctrL{} \and
      \LdruleLLXXctrR{} \and
      \LdruleLLXXexL{} \and
      \LdruleLLXXexR{} \and
      \LdruleLLXXILexL{} \and
      \LdruleLLXXCLexL{}
    \end{mathpar}
  \end{mdframed}
  \caption{Inference Rules for BiLNL Logic: Structural Rules}
  \label{fig:ifr-CL}
\end{figure}

\begin{figure}
  \begin{mdframed}
    \begin{mathpar}
      \LdruleLLXXid{} \and
      \LdruleLLXXcut{} \and
      \LdruleLLXXILcut{} \and
      \LdruleLLXXCLcut{} 
    \end{mathpar}
  \end{mdframed}
  \caption{Inference Rules for BiLNL Logic: Identity and Cut Rules}
  \label{fig:ifr-CL}
\end{figure}

\begin{figure}
  \begin{mdframed}
    \begin{mathpar}
      \LdruleLLXXIL{} \and
      \LdruleLLXXIR{} \and
      \LdruleLLXXcL{} \and
      \LdruleLLXXtL{} \and
      \LdruleLLXXtR{} 
    \end{mathpar}
  \end{mdframed}
  \caption{Inference Rules for BiLNL Logic: Conjunction and Tensor Rules}
  \label{fig:ifr-CL}
\end{figure}

\begin{figure}
  \begin{mdframed}
    \begin{mathpar}
      \LdruleLLXXJL{} \and
      \LdruleLLXXJR{} \and
      \LdruleLLXXdR{} \and
      \LdruleLLXXpL{} \and
      \LdruleLLXXpR{} 
    \end{mathpar}
  \end{mdframed}
  \caption{Inference Rules for BiLNL Logic: Disjunction and Par Rules}
  \label{fig:ifr-CL}
\end{figure}

\begin{figure}
  \begin{mdframed}
    \begin{mathpar}
      \LdruleLLXXiL{} \and
      \LdruleLLXXiR{} \and
      \LdruleLLXXILiL{} 
    \end{mathpar}
  \end{mdframed}
  \caption{Inference Rules for BiLNL Logic: Implication Rules}
  \label{fig:ifr-CL}
\end{figure}

\begin{figure}
  \begin{mdframed}
    \begin{mathpar}
      \LdruleLLXXsL{} \and
      \LdruleLLXXsR{} \and
      \LdruleLLXXCLsR{} 
    \end{mathpar}
  \end{mdframed}
  \caption{Inference Rules for BiLNL Logic: Co-implication Rules}
  \label{fig:ifr-CL}
\end{figure}

\begin{figure}
  \begin{mdframed}
    \begin{mathpar}
      \LdruleLLXXfL{} \and
      \LdruleLLXXfR{} \and
      \LdruleLLXXgL{} \and
      \LdruleLLXXjL{} \and
      \LdruleLLXXjR{} \and
      \LdruleLLXXhR{} 
    \end{mathpar}
  \end{mdframed}
  \caption{Inference Rules for BiLNL Logic: Adjoint Functors Rules}
  \label{fig:ifr-CL}
\end{figure}

\begin{figure}
  \begin{mdframed}
    \begin{mathpar}
      \Ldrulerl{} \and
      \Ldrulets{} \and
      \Ldrulecut{} \and
      \Ldruleid{} \and
      \LdrulemL{} \and
      \LdrulemR{} \and
      \LdruletL{} \and
      \LdruletR{} \and
      \LdrulefL{} \and
      \LdrulefR{} \and
      \LdruleaL{} \and
      \LdruleaR{} \and
      \LdruledL{} \and
      \LdruledR{} \and
      \LdruleiL{} \and
      \LdruleiR{} \and
      \LdrulesL{} \and
      \LdrulesR{} 
    \end{mathpar}
  \end{mdframed}
  \caption{Inference Rules for L}
  \label{fig:ifr-L}
\end{figure}


\bibliographystyle{plainurl} \bibliography{ref}

\end{document}

%%% Local Variables: 
%%% mode: latex
%%% TeX-master: t
%%% End: 

